\documentclass[a4paper, 12pt]{article}
\usepackage[italian]{babel}
\usepackage[colorlinks=true, linkcolor=blue, urlcolor=cyan]{hyperref}
\usepackage[utf8]{inputenc}
\usepackage{enumerate}
\title{Setup Github e VSCode per \LaTeX} 
\date{\today}
\author{Telemaco Pezzoni}

\begin{document}
\maketitle
\section{Captatio benevolentiae}
Documento scritto forse troppo in fretta per mostrare praticamente come configurare VSCode e Github per lavorare assieme in \LaTeX.
Non so quanto sia chiaro quindi quando installate le robe ovviamente chiedete pure per qualsiasi cosa (alla peggio chiediamo a chatgpt).
In ogni caso fatemi sapere se funziona tutto o meno, per chi non avesse sbatti (o spazio sul computer) possiamo organizzarci per setuppare anche Papeeria online.
Per la parte su come fare il commit e push (le modifiche) ho fatto il video (questo).
Questo documento lo trovate aggiornato nella repository alla cartella \textbf{src} con il nome \textbf{Istruzioni.tex}.
Il pdf generato lo trovate nella sottocartella \textbf{out}.
\section{Primi passaggi}
\begin{enumerate}[1.]
    \item Creare un account su \href{https://github.com/signup}{\textbf{Github}};
    \item Aggiungere ai preferiti la \href{https://github.com/Basicnickname/Progetto-Lab-3.git}{repository} e farsi aggiungere come collaboratore;
\end{enumerate}
\section{Installazione software}
\begin{enumerate}[1.]
    \item Scaricare \href{https://code.visualstudio.com/}{\textbf{VSCode}} e installarlo;
    \item Aprire VSCode e installare le estensioni (icona nella barra a sinistra oppure ctrl+shift+x) \textbf{LaTeX Workshop}, \textbf{GitHub Pull Requests}, \textbf{GitHub Copilot} e \textbf{Colab};
    \item Scaricare e installare \href{https://www.tug.org/texlive/windows.html#install}{texlive} (consiglio dopo aver aperto l'installer di andare nelle impostazioni avanzate e selezionare l'installazione media);
    \item Installare \href{https://git-scm.com/download/win}{Git} (consiglio di lasciare tutte le impostazioni di default);
    \item Riavviare il computer;
\end{enumerate}

\section{Configurazione}    
\begin{enumerate}[1.]
    \item Aprire VSCode premere ctrl+shift+p e digitare Preferences: Open User Settings (JSON), aprire il file e copiarci all'interno il contenuto del file \href{https://github.com/Basicnickname/Progetto-Lab-3/blob/main/json}{json};
    \item Aprire GIT Bash (dal menù start) e digitare i seguenti comandi: git config --global user.email "youremail@abc.example" e git config --global user.name "Your.Name";
    \item Aprire VSCode, premere ctrl+shift+p e digitare Git: Clone, incollare \href{https://github.com/Basicnickname/Progetto-Lab-3.git}{url} e scegliere la cartella di destinazione (consiglio il desktop);
\end{enumerate}

\end{document}