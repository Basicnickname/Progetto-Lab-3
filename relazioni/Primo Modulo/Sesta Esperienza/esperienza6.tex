\documentclass[a4paper,12pt]{article}

% --- Pacchetti Fondamentali ---
\usepackage[utf8]{inputenc} % Codifica caratteri
\usepackage[T1]{fontenc}    % Codifica font
\usepackage[italian]{babel} % Lingua italiana
\usepackage{geometry}       % Gestione margini
\usepackage{parskip}        % Gestione spazi tra paragrafi
\usepackage{hyperref}
\geometry{a4paper, margin=2.5cm}

% --- Pacchetti Matematica e Fisica ---
\usepackage{amsmath, amssymb} % Simboli matematici
\usepackage{siunitx}          % Gestione unità di misura e numeri (FONDAMENTALE)
\sisetup{
    output-decimal-marker = {.}, % Punto come separatore decimale
    separate-uncertainty = true, % Scrive errori come (Valore +/- Errore)
    per-mode = power,            % Usa / per le unità (m/s)
    inter-unit-product = \ensuremath{\cdot} % Per scrivere le unità con il punto (N·m)
}

% --- Pacchetti Immagini e Tabelle ---
\usepackage{graphicx} % Per inserire PNG/JPG
\usepackage{float}    % Per forzare la posizione delle immagini (H)
\usepackage{booktabs} % Per tabelle professionali (\toprule, \midrule)
\usepackage{caption}  % Per personalizzare le didascalie
\usepackage{subcaption}
\usepackage[font=small, labelfont=bf]{caption}
\usepackage[font=small, labelfont=bf]{subcaption}

% --- Dati Intestazione ---
\title{\textbf{Diffrazione da fenditure}}
\author{Filippo Audisio, Cataldo Insalaco, Telemaco Pezzoni}
\date{\today}

\begin{document}

\maketitle

% -------------------------------------------------------------------
\section{Obiettivo dell'esperienza}
    L'obiettivo è quello di studiare il fenomeno della diffrazione da singola fenditura e da doppia fenditura. 
    In particolare per la singola fenditura calcolare la sua larghezza, mentre per la doppia calcolare la larghezza delle singole fenditure e la distanza tra queste.
    
    Infine, usando sempre il fenomeno della diffrazione, misurare lo spessore di un capello.

% -------------------------------------------------------------------
\section{Materiali e Metodi}
    \subsection{Strumentazione}
        Strumenti e materiali utilizzati:
        \begin{itemize}
            \item Fenditure singole o doppie di diverse misure
            \item Laser rosso ($\lambda$=635.7 \unit{nm}) e verde ($\lambda$=531.9 \unit{nm})
            \item Metro
            \item Generatore di tensione continua
            \item Carta millimetrata
            \item Binario e supporti ottici
        \end{itemize}

    \subsection{Procedura sperimentale}
        Un laser e una fenditura (singola o doppia) sono stati inseriti in due supporti ottici. Il laser è stato collegato al generatore di tensione continua.
        La carta millimetrata è stata attaccata su una parete e il binario, su cui si trovano laser e fenditura, è stata posta a una distanza L maggiore di 1 m dalla parete.
        Quando il laser è stato acceso, si è cercato di illuminare la fenditura, centrandola il più possibile.
        
        Per la singola fenditura è stata misurata la distanza x dei minimi di diffrazione dal centro della figura luminosa che compare sulla carta millimetrata. 
        Il procedimento è stato per diversi valori di b, spessore della fenditura.

        Per la doppia fenditura sono stati misurati sia i minimi di diffrazione che di interferenza.
        La diffrazione crea una figura che ha un massimo centrale più largo e più luminoso rispetto agli altri massimi, mentre per l'interferenza i massimi hanno tutti la stessa inetnsità e sono piùravvicinati tra loro.
        Anche in questo caso si è svolto il procedimento per diversi valori di b, spessore della fenditura, e diversi valori di d, distanza tra le fenditure.

% -------------------------------------------------------------------
\section{Analisi dei dati e grafici}
%T: per mettere tutti i dati unire i grafici che descrivono lo stesso fenomeno specificando bene la legenda per le differenze, circa le considerazioni estetiche come per la 5, mettere le lettere greche negli assi sarebbe figo

    \subsection{Tabelle Risultati Fit}
        \subsubsection{Tabelle singola fenditura}
            Per fare il fit dei dati si usa la formula $\sin(\theta) = \frac{\lambda}{b}p$, con p l'ordine del minimo di diffrazione: si ha quindi una dipendenza lineare di $\sin(\theta)$ da p.
            Però sono state misurate le distanze x a cui si trovano i minimi, non l'angolo, ma la fenditura si trova a una distanza molto maggiore rispetto alle x misurate, 
            quindi si può fare l'approssimazione per piccoli angoli $\sin(\theta) \simeq \tan(\theta) = \frac{x}{L}$, dove L è la distanza della fenditura dalla parete dove si trova la carta millimetrata.
            \begin{table}[H]
                \centering
                \caption{Misure di b con il laser verde}
                \label{tab:dati_b_singola_verde}
                \begin{tabular}{
                    S[table-format=1.3]  % Formato per il reale valore di b
                    S[table-format=1.3]  % Formato per il valore b
                    S[table-format=1.3]  % Formato per sigma
                    S[table-format=2.2]  % Formato per l'errore relativo
                }
                    \toprule
                    {Valore reale b [\unit{\milli\meter}]} & {Valore fit b [\unit{\milli\meter}]} & {$\sigma$} & {Errore Relativo [\%]} \\
                    \midrule
                    0.1 & 0.109 & 0.004 & 3.61 \\
                    0.4 & 0.449 & 0.024 & 5.36 \\
                    0.8 & 0.901 & 0.058 & 6.46 \\
                    \bottomrule
                \end{tabular}
            \end{table}

            \begin{table}[H]
                \centering
                \caption{Misure di b con il laser rosso}
                \label{tab:dati_b_singola_rosso}
                \begin{tabular}{
                    S[table-format=1.3]  % Formato per il reale valore di b
                    S[table-format=1.3]  % Formato per il valore b
                    S[table-format=1.3]  % Formato per sigma
                    S[table-format=2.2]  % Formato per l'errore relativo
                }
                    \toprule
                    {Valore reale b [\unit{\milli\meter}]} & {Valore fit b [\unit{\milli\meter}]} & {$\sigma$} & {Errore Relativo [\%]} \\
                    \midrule
                    0.1 & 0.104 & 0.018 & 17.05 \\
                    0.4 & 0.479 & 0.027 & 5.59 \\
                    0.8 & 0.800 & 0.029 & 3.72 \\
                    \bottomrule
                \end{tabular}
            \end{table}

        \subsubsection{Tabelle doppia fenditura}
            Come per la singola fenditura usando la formula $\sin(\theta) = \frac{\lambda}{b}p$ per fare il fit dei dati si può ottenere un valore per il parametro b. 
            Usando $\sin(\theta) = \frac{\lambda}{d}(m+\frac{1}{2})$ si ha una stima del valore del parametro d.

            \begin{table}[H]
                \centering
                \caption{Misure di b e d con il laser verde}
                \label{tab:dati_b_d_doppia_verde}
                \begin{tabular}{
                    S[table-format=1.3]  % Formato per il reale valore di b
                    S[table-format=1.3]  % Formato per il valore b
                    S[table-format=1.3]  % Formato per l'errore
                    S[table-format=2.2]  % Formato per l'errore relativo
                }
                    \toprule
                    {Valore reale b [\unit{\milli\meter}]} & {Valore fit b [\unit{\milli\meter}]} & {$\sigma$} & {Errore Relativo [\%]} \\
                    \midrule
                    0.1 & 0.081 & 0.002 & 2.78 \\
                    0.15 & 0.112 & 0.012 & 10.67 \\
                    0.15 & 0.137 & 0.008 & 6.02 \\
                    \bottomrule
                \end{tabular}
                \begin{tabular}{
                    S[table-format=1.3]  % Formato per il reale valore di d
                    S[table-format=1.3]  % Formato per il valore d
                    S[table-format=1.3]  % Formato per l'errore
                    S[table-format=2.2]  % Formato per l'errore relativo
                }
                    \toprule
                    {Valore reale d [\unit{\milli\meter}]} & {Valore fit d [\unit{\milli\meter}]} & {$\sigma$} & {Errore Relativo [\%]} \\
                    \midrule
                    0.3 & 0.295 & 0.011 & 3.84 \\
                    0.25 & 0.244 & 0.004 & 1.51 \\
                    0.5 & 0.414 & 0.036 & 8.58 \\
                    \bottomrule
                \end{tabular}
            \end{table}

            \begin{table}[H]
                \centering
                \caption{Misure di b e d con il laser rosso}
                \label{tab:dati_b_d_doppia_rosso}
                \begin{tabular}{
                    S[table-format=1.3]  % Formato per il reale valore di b
                    S[table-format=1.3]  % Formato per il valore b
                    S[table-format=1.3]  % Formato per l'errore
                    S[table-format=2.2]  % Formato per l'errore relativo
                }
                    \toprule
                    {Valore reale b [\unit{\milli\meter}]} & {Valore fit b [\unit{\milli\meter}]} & {$\sigma$} & {Errore Relativo [\%]} \\
                    \midrule
                    0.1 & 0.088 & 0.005 & 6.04 \\
                    0.15 & 0.082 & 0.010 & 12.12 \\
                    0.15 & 0.113 & 0.010 & 8.87 \\
                    \bottomrule
                \end{tabular}
                \begin{tabular}{
                    S[table-format=1.3]  % Formato per il reale valore di d
                    S[table-format=1.3]  % Formato per il valore d
                    S[table-format=1.3]  % Formato per l'errore
                    S[table-format=2.2]  % Formato per l'errore relativo
                }
                    \toprule
                    {Valore reale d [\unit{\milli\meter}]} & {Valore fit d [\unit{\milli\meter}]} & {$\sigma$} & {Errore Relativo [\%]} \\
                    \midrule
                    0.3 & 0.297 & 0.007 & 2.43 \\
                    0.3 & 0.317 & 0.010 & 3.09 \\
                    0.25 & 0.272 & 0.010 & 3.63 \\
                    \bottomrule
                \end{tabular}
            \end{table}
        
        \subsubsection{Diffrazione da un capello}
            Questo caso è diverso dai precedenti: invece che una fenditura colpita da un fascio luminoso si ha un corpo opaco, un capello. 
            Le equazioni che descrivono questa situazione, però sono le stesse utilizzate in precedenza, cioè $\sin(\theta) = \frac{\lambda}{b}p$ dove in questo caso b rappresenta lo spessore del corpo opaco. 
            Il laser utilizzato è quello verde si può allora fare il fit della funzione e si orriene un valore per il parametro b.
            \begin{table}[H]
                \centering
                \caption{Diffrazione da un capello}
                \label{tab:dati_capello}
                \begin{tabular}{
                    S[table-format=2.2]  % Formato per il valore b
                    S[table-format=1.2]  % Formato per l'errore
                    S[table-format=2.2]  % Formato per l'errore relativo
                }
                \toprule
                {Valore fit b [\unit{\micro\meter}]} & {$\sigma$} & {Errore Relativo [\%]} \\
                \midrule
                67.84 & 2.63 & 3.88 \\
                \bottomrule
                \end{tabular}
            \end{table}


    \subsection{Grafici e curve di regressione}
        Di seguito sono riportati i grafici di confronto tra i dati sperimentali e le curve teoriche usando le $b$ e le $d$ trovate nei fit dei dati.
        \subsubsection{Singola fenditura}
            \begin{figure}[H]
                \centering
                \begin{subfigure}{0.48\textwidth}
                    \includegraphics[width=\linewidth]{Singola_fenditura_verde.png}
                    \caption{Laser verde}
                    \label{fig:b_singola_verde}
                \end{subfigure}
                \begin{subfigure}{0.48\textwidth}
                    \includegraphics[width=\linewidth]{Singola_fenditura_rosso.png}
                    \caption{Laser rosso}
                    \label{fig:b_singola_rosso}
                \end{subfigure}
                \caption{$\sin(\theta)$ in funzione dell'ordine del minimo di diffrazione osservato}
                \label{fig:singola}
            \end{figure}
        
        \subsubsection{Doppia fenditura}
            \subsubsection{Laser verde}
                \begin{figure}[H]
                    \centering
                    \includegraphics[width=0.9\textwidth]{Doppia_fenditura_verde.png}
                    \caption{A sinistra: $\sin(\theta)$ in funzione dell'ordine del minimo di diffrazione osservato. A destra: $\sin(\phi)$ in funzione del minimo di interferenza osservato.}
                    \label{fig:doppia_verde}
                \end{figure}
            \subsubsection{Laser rosso}
                \begin{figure}[H]
                    \centering
                    \includegraphics[width=0.9\textwidth]{Doppia_fenditura_rosso.png}
                    \caption{A sinistra: $\sin(\theta)$ in funzione dell'ordine del minimo di diffrazione osservato. A destra: $\sin(\phi)$ in funzione del minimo di interferenza osservato.}
                    \label{fig:doppia_rosso}
                \end{figure}
        \subsubsection{Capello}
            \begin{figure}[H]
                \centering
                \includegraphics[width=0.49\textwidth]{Capello.png}
                \caption{Diffrazione da un capello}
                \label{fig:capello}
            \end{figure}

% -------------------------------------------------------------------

\section{Conclusioni}
Dai grafici si osserva la dipendenza lineare di $\sin(\theta)$ da p, ordine del minimo di diffrazione o interferenza.
I dati per b e d ottenuti dai fit sono quasi tutti coerenti con quelli reali, a eccezione di uno, in cui il b dal fit viene $(0.082\pm0.006)$\unit{\milli\meter} mentre quello reale è di $0.15$ mm, ma questo potrebbe essere dovuto al fatto che in quel caso siamo riusciti a prendere solo 3 dati per quella misura.

Per il fit del capello lo spessore viene di $(67.84\pm1.32)$\unit{\micro\meter} che è una misura sensata; infatti normalmente lo spessore dei capelli si aggira tra i 60 e i 100 \unit{\micro\meter}
\end{document}