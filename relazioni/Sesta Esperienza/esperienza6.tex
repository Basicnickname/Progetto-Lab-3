\documentclass[a4paper,12pt]{article}

% --- Pacchetti Fondamentali ---
\usepackage[utf8]{inputenc} % Codifica caratteri
\usepackage[T1]{fontenc}    % Codifica font
\usepackage[italian]{babel} % Lingua italiana
\usepackage{geometry}       % Gestione margini
\geometry{a4paper, margin=2.5cm}

% --- Pacchetti Matematica e Fisica ---
\usepackage{amsmath, amssymb} % Simboli matematici
\usepackage{siunitx}          % Gestione unità di misura e numeri (FONDAMENTALE)
\sisetup{
    output-decimal-marker = {.}, % Punto come separatore decimale
    separate-uncertainty = true, % Scrive errori come (Valore +/- Errore)
    per-mode = symbol            % Usa / per le unità (m/s)
}

% --- Pacchetti Immagini e Tabelle ---
\usepackage{graphicx} % Per inserire PNG/JPG
\usepackage{float}    % Per forzare la posizione delle immagini (H)
\usepackage{booktabs} % Per tabelle professionali (\toprule, \midrule)
\usepackage{caption}  % Per personalizzare le didascalie

% --- Dati Intestazione ---
\title{\textbf{Caratterizzazione elettrica e ottica dei LED}}
\author{Filippo Audisio, Cataldo Insalaco, Telemaco Pezzoni}
\date{\today}

\begin{document}

\maketitle

% -------------------------------------------------------------------
\section{Obiettivo dell'esperienza}
L'obiettivo è quello di studiare il fenomeno della diffrazione da singola fenditura e da doppia fenditura. 
In particolare per la singola fenditura calcolare la sua larghezza, mentre per la doppia calcolare la larghezza delle singole fenditure e la distanza tra queste.

% -------------------------------------------------------------------
\section{Materiali e Metodi}

\subsection{Strumentazione utilizzata}
Strumenti e materiali utilizzati:
\begin{itemize}
    \item Fenditure singole o doppie di diverse misure
    \item Laser verde e rosso
    \item Metro (Sensibilità: \SI{0.01}{\meter})
    \item Generatore di funzione ? (o il generatore in continua)
    \item Carta millimetrata
    \item Binario e supporti ottici
\end{itemize}

\subsection{Procedura sperimentale}
Inserire il laser e una fenditura (singola o doppia) nei supporti e collegare il laser al generatore in continua (?) (generatore di funzione).
Attaccare la carta millimetrata su una parete e porre il binario a una distanza L maggiore di 1 m da questa.
Accendere il laser e illuminare la fenditura.
Misurare la distanza x dei minimi di diffrazione, anche di interferenza nel caso della doppia fenditura, dal centro della figura luminosa che compare sulla carta millimetrata.
\\
L'equazione che descrive la diffrazione è $\sin(\theta) = \frac{\lambda}{b}p$ con p l'ordine del minimo di diffrazione ($\sin(\theta) = \frac{\lambda}{b}(m+\frac{1}{2})$ per l'interferenza): si ha quindi una dipendenza lineare di $\sin(\theta)$ da p.
La fenditura trovandosi a una distanza molto maggiore rispetto alla lunghezza d'onda della luce incidente sulla fenditura, si può approssimare $\sin(\theta) \simeq \tan(\theta) = \frac{x}{L}$.

% -------------------------------------------------------------------
\section{Dati sperimentali e Analisi}

\subsection{Grafici dei dati ottenuti}
Di seguito sono riportati i grafici dei dati ottenuti sia per la singola che per la doppia fenditura e specificando anche il colore del laser utilizzato.

\subsection{Singola fenditura}
\begin{figure}[H]
    \centering
    \includegraphics[width=0.49\textwidth]{1.Verde.b=0,1.png}
    \includegraphics[width=0.49\textwidth]{1.Verde.b=0,4.png}
\end{figure}
\begin{figure}[H]
    \centering
    \includegraphics[width=0.49\textwidth]{1.Verde.b=0,8.png}
    \includegraphics[width=0.49\textwidth]{1.Rosso.b=0,1.png}
\end{figure}
\begin{figure}[H]
    \centering
    \includegraphics[width=0.49\textwidth]{1.Rosso.b=0,4.png}
    \includegraphics[width=0.49\textwidth]{1.Rosso.b=0,8.png}
\end{figure}

\subsection{Doppia fenditura}
\begin{figure}[H]
    \centering
    \includegraphics[width=0.49\textwidth]{2.Verde.b=0,1.png}
    \includegraphics[width=0.49\textwidth]{2.d=0,25.Verde.b=0,15.png}
\end{figure}
\begin{figure}[H]
    \centering
    \includegraphics[width=0.49\textwidth]{2.d=0,5.Verde.b=0,15.png}
    \includegraphics[width=0.49\textwidth]{2.Verde.d=0,3.png}
\end{figure}
\begin{figure}[H]
    \centering
    \includegraphics[width=0.49\textwidth]{2.Verde.d=0,25.png}
    \includegraphics[width=0.49\textwidth]{2.Verde.d=0,5.png}
\end{figure}
\begin{figure}[H]
    \centering
    \includegraphics[width=0.49\textwidth]{2.Rosso.b=0,1.png}
    \includegraphics[width=0.49\textwidth]{2.d=0,3.Rosso.b=0,15.png}
\end{figure}
\begin{figure}[H]
    \centering
    \includegraphics[width=0.49\textwidth]{2.d=0,25.Rosso.b=0,15.png}
    \includegraphics[width=0.49\textwidth]{2.b=0,1.Rosso.d=0,3.png}
\end{figure}
\begin{figure}[H]
    \centering
    \includegraphics[width=0.49\textwidth]{2.b=0,15.Rosso.d=0,3.png}
    \includegraphics[width=0.49\textwidth]{2.Rosso.d=0,25.png}
\end{figure}

\subsection{Tabelle Risultati Fit}
\subsection{Tabelle singola fenditura}
Usando la formula $\sin(\theta) = \frac{\lambda}{b}p$ per fare il fit dei dati si può ottenere un valore per il parametro b. 
\begin{table}[H]
    \centering
    \caption{Misure di b con il laser verde}
    \label{tab:dati_b_singola_verde}
    \begin{tabular}{
        S[table-format=1.3]  % Formato per il reale valore di b
        S[table-format=1.3]  % Formato per il valore b (2 cifre intere, 1 decimale)
        S[table-format=1.3]  % Formato per l'errore
        S[table-format=2.2]  % Formato per l'errore relativo
    }
        \toprule
        {Valore reale b [\unit{\milli\meter}]} & {Valore fit b [\unit{\milli\meter}]} & {Errore Assoluto [\unit{\milli\meter}]} & {Errore Relativo [\%]} \\
        \midrule
        0.1 & 0.109 & 0.004 & 3.61 \\
        0.4 & 0.449 & 0.024 & 5.36 \\
        0.8 & 0.901 & 0.058 & 6.46 \\
        \bottomrule
    \end{tabular}
\end{table}

\begin{table}[H]
    \centering
    \caption{Misure di b con il laser rosso}
    \label{tab:dati_b_singola_rosso}
    \begin{tabular}{
        S[table-format=1.3]  % Formato per il reale valore di b
        S[table-format=1.3]  % Formato per il valore b (2 cifre intere, 1 decimale)
        S[table-format=1.3]  % Formato per l'errore
        S[table-format=2.2]  % Formato per l'errore relativo
    }
        \toprule
        {Valore reale b [\unit{\milli\meter}]} & {Valore fit b [\unit{\milli\meter}]} & {Errore Assoluto [\unit{\milli\meter}]} & {Errore Relativo [\%]} \\
        \midrule
        0.1 & 0.104 & 0.018 & 17.05 \\
        0.4 & 0.479 & 0.027 & 5.59 \\
        0.8 & 0.800 & 0.298 & 3.72 \\
        \bottomrule
    \end{tabular}
\end{table}

\subsection{Tabelle doppia fenditura}
Come per la singola fenditura usando la formula $\sin(\theta) = \frac{\lambda}{b}p$ per fare il fit dei dati si può ottenere un valore per il parametro b. 
Usando $\sin(\theta) = \frac{\lambda}{d}(m+\frac{1}{2})$ si ha una stima del valore del parametro d.

\begin{table}[H]
    \centering
    \caption{Misure di b e d con il laser verde}
    \label{tab:dati_b_d_doppia_verde}
    \begin{tabular}{
        S[table-format=1.3]  % Formato per il reale valore di b
        S[table-format=1.3]  % Formato per il valore b (2 cifre intere, 1 decimale)
        S[table-format=1.3]  % Formato per l'errore
        S[table-format=2.2]  % Formato per l'errore relativo
    }
        \toprule
        {Valore reale b [\unit{\milli\meter}]} & {Valore fit b [\unit{\milli\meter}]} & {Errore Assoluto [\unit{\milli\meter}]} & {Errore Relativo [\%]} \\
        \midrule
        0.1 & 0.081 & 0.002 & 2.78 \\
        0.15 & 0.112 & 0.012 & 10.67 \\
        0.15 & 0.137 & 0.008 & 6.02 \\
        \bottomrule
    \end{tabular}
    \begin{tabular}{
        S[table-format=1.3]  % Formato per il reale valore di d
        S[table-format=1.3]  % Formato per il valore d (2 cifre intere, 1 decimale)
        S[table-format=1.3]  % Formato per l'errore
        S[table-format=2.2]  % Formato per l'errore relativo
    }
        \toprule
        {Valore reale d [\unit{\milli\meter}]} & {Valore fit d [\unit{\milli\meter}]} & {Errore Assoluto [\unit{\milli\meter}]} & {Errore Relativo [\%]} \\
        \midrule
        0.3 & 0.295 & 0.011 & 3.84 \\
        0.25 & 0.244 & 0.004 & 1.51 \\
        0.5 & 0.414 & 0.036 & 8.58 \\
        \bottomrule
    \end{tabular}
\end{table}

\begin{table}[H]
    \centering
    \caption{Misure di b e d con il laser rosso}
    \label{tab:dati_b_d_doppia_rosso}
    \begin{tabular}{
        S[table-format=1.3]  % Formato per il reale valore di b
        S[table-format=1.3]  % Formato per il valore b (2 cifre intere, 1 decimale)
        S[table-format=1.3]  % Formato per l'errore
        S[table-format=2.2]  % Formato per l'errore relativo
    }
        \toprule
        {Valore reale b [\unit{\milli\meter}]} & {Valore fit b [\unit{\milli\meter}]} & {Errore Assoluto [\unit{\milli\meter}]} & {Errore Relativo [\%]} \\
        \midrule
        0.1 & 0.088 & 0.005 & 6.04 \\
        0.15 & 0.082 & 0.010 & 12.12 \\
        0.15 & 0.113 & 0.010 & 8.87 \\
        \bottomrule
    \end{tabular}
    \begin{tabular}{
        S[table-format=1.3]  % Formato per il reale valore di d
        S[table-format=1.3]  % Formato per il valore d (2 cifre intere, 1 decimale)
        S[table-format=1.3]  % Formato per l'errore
        S[table-format=2.2]  % Formato per l'errore relativo
    }
        \toprule
        {Valore reale d [\unit{\milli\meter}]} & {Valore fit d [\unit{\milli\meter}]} & {Errore Assoluto [\unit{\milli\meter}]} & {Errore Relativo [\%]} \\
        \midrule
        0.3 & 0.297 & 0.007 & 2.43 \\
        0.3 & 0.317 & 0.010 & 3.09 \\
        0.25 & 0.272 & 0.010 & 3.63 \\
        \bottomrule
    \end{tabular}
\end{table}


\subsection{Plot}
Di seguito sono riportati i grafici di confronto tra i dati sperimentali e le curve disegnate usando le $b$ e le $d$ trovate nei fit dei dati.
\subsection{Singola fenditura}
\begin{figure}[H]
    \centering
    \includegraphics[width=0.49\textwidth]{Plot_1.Verde.b=0,1.png}
    \includegraphics[width=0.49\textwidth]{Plot_1.Verde.b=0,4.png}
\end{figure}
\begin{figure}[H]
    \centering
    \includegraphics[width=0.49\textwidth]{Plot_1.Verde.b=0,8.png}
    \includegraphics[width=0.49\textwidth]{Plot_1.Rosso.b=0,1.png}
\end{figure}
\begin{figure}[H]
    \centering
    \includegraphics[width=0.49\textwidth]{Plot_1.Rosso.b=0,4.png}
    \includegraphics[width=0.49\textwidth]{Plot_1.Rosso.b=0,8.png}
\end{figure}

\subsection{Doppia fenditura}
\subsection{Laser verde}
\begin{figure}[H]
    \centering
    \includegraphics[width=0.49\textwidth]{Plot_2.Verde.b=0,1.png}
    \includegraphics[width=0.49\textwidth]{Plot_2.d=0,25.Verde.b=0,15.png}
\end{figure}
\begin{figure}[H]
    \centering
    \includegraphics[width=0.49\textwidth]{Plot_2.d=0,5.Verde.b=0,15.png}
    \includegraphics[width=0.49\textwidth]{Plot_2.Verde.d=0,3.png}
\end{figure}
\begin{figure}[H]
    \centering
    \includegraphics[width=0.49\textwidth]{Plot_2.Verde.d=0,25.png}
    \includegraphics[width=0.49\textwidth]{Plot_2.Verde.d=0,5.png}
\end{figure}
\subsection{Laser rosso}
\begin{figure}[H]
    \centering
    \includegraphics[width=0.49\textwidth]{Plot_2.Rosso.b=0,1.png}
    \includegraphics[width=0.49\textwidth]{Plot_2.d=0,3.Rosso.b=0,15.png}
\end{figure}
\begin{figure}[H]
    \centering
    \includegraphics[width=0.49\textwidth]{Plot_2.d=0,25.Rosso.b=0,15.png}
    \includegraphics[width=0.49\textwidth]{Plot_2.b=0,1.Rosso.d=0,3.png}
\end{figure}
\begin{figure}[H]
    \centering
    \includegraphics[width=0.49\textwidth]{Plot_2.b=0,15.Rosso.d=0,3.png}
    \includegraphics[width=0.49\textwidth]{Plot_2.Rosso.d=0,25.png}
\end{figure}

% -------------------------------------------------------------------

\section{Diffrazione da un capello}
Questo caso è diverso dai precedenti: invece che una fenditura colpita da un fascio luminoso si ha un corpo opaco, un capello. 
Le equazioni che descrivono questa situazione sono le stesse utilizzate in precedenza, cioè $\sin(\theta) = \frac{\lambda}{b}p$ dove b rappresenta lo spessore del corpo opaco. 
\\
La procedura sperimentale è identica a quella precedente: collegare il laser al generatore di funzione (?) (generatore in continua), fissare il capello ben disteso a un supporto e illuminarlo con il laser, infine misurare la distanza x dei minimi di intensità dal centro della figura di diffrazione che si visualizza sulla carta millimetrata. 
Anche in questo caso si approssima la misura di x alla misura del seno dell'angolo a cui si trova il minimo.
\\
\subsection{Grafici e tabelle}
Sapendo che il laser utilizzato è quello verde ($\lambda = 531.9$ nm) si può fare il fit della funzione $\sin(\theta) = \frac{\lambda}{b}p$ e ottenere un valore per il parametro b, spessore del capello:
\begin{table}[H]
    \centering
    \caption{Diffrazione da un capello}
    \label{tab:dati_capello}
    \begin{tabular}{
        S[table-format=2.2]  % Formato per il valore b
        S[table-format=1.2]  % Formato per l'errore
        S[table-format=2.2]  % Formato per l'errore relativo
    }
        \toprule
        {Valore fit b [\unit{\micro\meter}]} & {Errore Assoluto [\unit{\micro\meter}]} & {Errore Relativo [\%]} \\
        \midrule
        67.84 & 2.63 & 3.88 \\
        \bottomrule
    \end{tabular}
\end{table}
    
\begin{figure}[H]
    \centering
    \includegraphics[width=0.49\textwidth]{Plot_Capello.png}
\end{figure}
% -------------------------------------------------------------------
\section{Conclusioni}
Dai grafici si osserva la dipendenza lineare di $\sin(\theta)$ da p, ordine del minimo di diffrazione o interferenza.
I dati per b e d ottenuti dai fit sono quasi tutti coerenti con quelli reali, a eccezione di uno, in cui il b dal fit viene $(0.082\pm0.010)$ mm mentre quello reale è di $0.15$ mm, ma questo potrebbe essere dovuto al fatto che in quel caso siamo riusciti a prendere solo 3 dati per quella misura.
Per il fit del capello lo spessore viene di $(67.84\pm2.63)\mu$m che è una misura sensata; infatti normalmente lo spessore dei capelli si aggira tra i 60 e i 100 $\mu$m
\end{document}