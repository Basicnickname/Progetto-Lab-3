\documentclass[a4paper,12pt]{article}

% --- Pacchetti Fondamentali ---
\usepackage[utf8]{inputenc} % Codifica caratteri
\usepackage[T1]{fontenc}    % Codifica font
\usepackage[italian]{babel} % Lingua italiana
\usepackage{geometry}       % Gestione margini
\geometry{a4paper, margin=2.5cm}

% --- Pacchetti Matematica e Fisica ---
\usepackage{amsmath, amssymb} % Simboli matematici
\usepackage{siunitx}          % Gestione unità di misura e numeri (FONDAMENTALE)
\sisetup{
    output-decimal-marker = {.}, % Punto come separatore decimale
    separate-uncertainty = true, % Scrive errori come (Valore +/- Errore)
    per-mode = symbol            % Usa / per le unità (m/s)
}

% --- Pacchetti Immagini e Tabelle ---
\usepackage{graphicx} % Per inserire PNG/JPG
\usepackage{float}    % Per forzare la posizione delle immagini (H)
\usepackage{booktabs} % Per tabelle professionali (\toprule, \midrule)
\usepackage{caption}  % Per personalizzare le didascalie

% --- Dati Intestazione ---
\title{\textbf{Diffrazione da fenditure}}
\author{Filippo Audisio, Cataldo Insalaco, Telemaco Pezzoni}
\date{\today}

\begin{document}

\maketitle

% -------------------------------------------------------------------
\section{Obiettivo dell'esperienza}
L'obiettivo è quello di studiare il fenomeno della diffrazione da singola fenditura e da doppia fenditura. In particolare per la singola fenditura calcolarne la larghezza e per la doppia calcolarne la larghezza e la distanza tra le fenditure.

% -------------------------------------------------------------------
\section{Materiali e Metodi}

\subsection{Strumentazione utilizzata}
Strumenti e materiali utilizzati:
\begin{itemize}
    \item Fenditure singole o doppie di diverse misure
    \item Laser verde e rosso
    \item Metro (Sensibilità: \SI{0.01}{m})
    \item Generatore di funzione
    \item Carta millimetrata
    \item Binari e supporti ottici
\end{itemize}

\subsection{Procedura sperimentale}
Descrivi qui come hai svolto l'esperimento, meglio un testo discorsivo senza formule. Specifica quali grandezze sono state misurate direttamente e quali indirettamente.
\textit{"Abbiamo posizionato il cervello sul tavolo e misurato qualcosa di straordinario..."}

% -------------------------------------------------------------------
\section{Dati sperimentali e Analisi}

In questa sezione riportiamo i dati raccolti e i grafici di fit.

\subsection{Grafici dati grezzi}
Di seguito sono riportati i grafici dei dati sperimentali grezzi. (Direi non tabelle sennò occupano tutta la pagina) %Sistemare nome del grafico e legenda su Colab, in alcuni fit sono presenti nel grafico anche i risultati del fit, toglierli sarebbe meglio.       


\subsection{Tabelle Risultati Fit}
fittando con la funzione $y = mx + q$ otteniamo i seguenti risultati:
\begin{table}[H]
    \centering
    \caption{Misure di corrente nel cervello se tocchi una presa elettrica.}
    \label{tab:dati_prima_misura}
    \begin{tabular}{
        l 
        S[table-format=2.1]  % Formato per il valore (2 cifre intere, 1 decimale)
        S[table-format=1.1]  % Formato per l'errore
        S[table-format=1.2]  % Formato per l'errore relativo
    }
        \toprule
        {Parametro} & {Valore [\unit{cm/s}]} & {Errore Assoluto [\unit{cm/s}]} & {Errore Relativo [\%]} \\
        \midrule
        $\Delta$ & 2.5 & 0.1 & 4.00 \\
        $\delta$ & 5.1 & 0.1 & 1.96 \\
        $\beta$ & 7.4 & 0.1 & 1.35 \\
        $\alpha$ & 10.0 & 0.2 & 2.00 \\
        \bottomrule
    \end{tabular}
\end{table}

\subsection{Plot}
Di seguito è riportato il grafico dei dati sperimentali con la curva di fit.



Molto bello
% -------------------------------------------------------------------
\section{Conclusioni}
\textbf{Esiti fisici:}
Molto bello torna tutto
\\
\textbf{Commenti:}
Discutere qui eventuali fonti di errore sistematico, la bontà del fit e possibili miglioramenti dell'esperimento. 
\end{document}