\documentclass[a4paper,12pt]{article}

% --- Pacchetti Fondamentali ---
\usepackage[utf8]{inputenc} % Codifica caratteri
\usepackage[T1]{fontenc}    % Codifica font
\usepackage[italian]{babel} % Lingua italiana
\usepackage{geometry}       % Gestione margini
\geometry{a4paper, margin=2.5cm}

% --- Pacchetti Matematica e Fisica ---
\usepackage{amsmath, amssymb} % Simboli matematici
\usepackage{siunitx}          % Gestione unità di misura e numeri (FONDAMENTALE)
\sisetup{
    output-decimal-marker = {.}, % Punto come separatore decimale
    separate-uncertainty = true, % Scrive errori come (Valore +/- Errore)
    per-mode = symbol            % Usa / per le unità (m/s)
}

% --- Pacchetti Immagini e Tabelle ---
\usepackage{graphicx} % Per inserire PNG/JPG
\usepackage{float}    % Per forzare la posizione delle immagini (H)
\usepackage{booktabs} % Per tabelle professionali (\toprule, \midrule)
\usepackage{caption}  % Per personalizzare le didascalie

% --- Dati Intestazione ---
\title{\textbf{Caratterizzazione elettrica e ottica LED}}
\author{Filippo Audisio, Cataldo Insalaco, Telemaco Pezzoni}
\date{\today}

\begin{document}

\maketitle

% -------------------------------------------------------------------
\section{Obiettivo dell'esperienza}
L'obiettivo dell'esperienza è studiare la caratterizzazione elettrica e ottica di alcuni LED di colori diversi, nello specifico:
\begin{itemize}
    \item Nella caratterizzazione elettrica verificare la legge esponenziale $I = I_0(e^\frac{qV}{nkT}-1)$ e calcolare il valore di $I_0$.
    \item Nella caratterizzazione ottica ricavare la lunghezza d'onda del massimo della luce emessa dai LED.
\end{itemize}

% -------------------------------------------------------------------
\section{Materiali e Metodi}

\subsection{Strumentazione utilizzata}
\subsection{Strumentazione caratterizzazione elettrica}
\begin{itemize}
    \item LED colore bianco, blu, giallo, rosso, verde e infrarosso
    \item Fotodiodo
    \item Due multimetri
    \item Resistenza da $511 \Omega$
    \item Cavi a banana
    \item Alimentatore in continua
\end{itemize}

\subsection{Strumentazione caratterizzazione ottica}
\begin{itemize}
    \item LED colore bianco, blu, giallo, rosso, verde e infrarosso
    \item Fotodiodo
    \item Due multimetri
    \item Cavi a banana
    \item Alimentatore in continua
    \item Reticolo (1000 linee/mm)
    \item Scatola condizionamento segnale
    \item Binario
    \item Braccio rotante
    \item Tre supporti
    \item Due lenti
    \item Foglio con scala angolare
\end{itemize}

\subsection{Procedura sperimentale}
\subsection{Caratterizzazione elettrica}
Collegare in serie un multimetro, usato come amperometro, il LED e la resistenza all'alimentatore in continua. 
Collegare in parallelo al LED l'altro multimetro, utilizzandolo come voltmetro.
Variando la corrente nell'alimentatore alimentare il LED registrando le coppie di valori tensione-corrente lette sui multimetri.
Prendere nota dei valori di corrente per i quali si ha l'apparire di emissione luminosa da parte del LED.

\subsection{Caratterizzazione ottica}
Inserire il LED e il reticolo sui supporti ottici presenti sul binario, mentre il fotodiodo sul supporto che si trova sul braccio rotante, a una distanza di 39 cm dal reticolo.
Collegare in serie un multimetro, utilizzato come amperometro, e il LED all'alimentatore in continua
Collegare il fotodiodo all'altro multimetro, utilizzato come voltmetro, tramite la scatola di condizionameto del segnale, che permette di avere una risposta lineare del fotodiodo.
Allineare il braccio rotante e il foglio con la scala angolare cercando prima l'ordine zero: si deve osservare un massimo di risposta in tensione del fotodiodo quando LED, reticolo e fotodiodo sono allineati.
Successivamente verificare che ruotando l'asta l'estremità del braccio rotante continui a scorrere smepre sulla linea dei 39 cm presente sul foglio con scala angolare.
Far scorrere l'asta rotante e prendere nota dei valori in tensione in corrispondenza delle diverse lunghezze d'onda presenti sul foglio.

% -------------------------------------------------------------------
\section{Dati sperimentali e Analisi}
\subsection{Grafici dati caratterizzazione elettrica}
\begin{figure}[H]
    \centering
    \includegraphics[width=0.55\textwidth]{LED Giallo.png}
    \label{grafico_exp_giallo}
\end{figure}
\begin{figure}[H]
    \centering
    \includegraphics[width=0.55\textwidth]{LED Rosso.png}
    \label{grafico_exp_rosso}
\end{figure}
\begin{figure}[H]
    \centering
    \includegraphics[width=0.55\textwidth]{LED Blu.png}
    \label{grafico_exp_blu}
\end{figure}
\begin{figure}[H]
    \centering
    \includegraphics[width=0.55\textwidth]{LED Verde.png}
    \label{grafico_exp_verde}
\end{figure}
\begin{figure}[H]
    \centering
    \includegraphics[width=0.55\textwidth]{LED Bianco.png}
    \label{grafico_exp_bianco}
\end{figure}
\begin{figure}[H]
    \centering
    \includegraphics[width=0.55\textwidth]{LED Infrarosso.png}
    \label{grafico_exp_infrarosso}
\end{figure}
\begin{figure}[H]
    \centering
    \includegraphics[width=0.55\textwidth]{Fotodiodo.png}
    \label{grafico_exp_fotodiodo}
\end{figure}

\subsection{Grafici caratterizzazione ottica}
\begin{figure}[H]
    \centering
    \includegraphics[width=0.55\textwidth]{LED Giallo ottica.png}
    \label{grafico_ott_giallo}
\end{figure}
\begin{figure}[H]
    \centering
    \includegraphics[width=0.55\textwidth]{LED Rosso ottica.png}
    \label{grafico_ott_rosso}
\end{figure}
\begin{figure}[H]
    \centering
    \includegraphics[width=0.55\textwidth]{LED Blu ottica.png}
    \label{grafico_ott_blu}
\end{figure}
\begin{figure}[H]
    \centering
    \includegraphics[width=0.55\textwidth]{LED Verde ottica.png}
    \label{grafico_ott_verde}
\end{figure}
\begin{figure}[H]
    \centering
    \includegraphics[width=0.55\textwidth]{LED Bianco ottica.png}
    \label{grafico_ott_bianco}
\end{figure}
\begin{figure}[H]
    \centering
    \includegraphics[width=0.55\textwidth]{LED Infrarosso ottica.png}
    \label{grafico_ott_infrarosso}
\end{figure}

\subsection{Tabelle Risultati Fit}
Usando come funzione di fit $I = I_0(e^\frac{qV}{nkT}-1)$ abbiamo ottenuto come valori di $I_0$:
\begin{table}[H]
    \centering
    \label{tab:I0_LED}
    \begin{tabular}{
        l 
        S[scientific-notation = true, table-format=1.2e2]   % Formato per il valore
        S[scientific-notation = true, table-format=1.2e2]   % Formato per l'errore
        S[table-format=3.2] % Formato per l'errore relativo
    }
        \toprule
        {LED} & {Valore $I_0$ [\unit{A}]} & {Errore Assoluto $I_0$ [\unit{A}]} & {Errore Relativo $I_0$ [\%]} \\
        \midrule
        Giallo & 3.28e-12 & 8.44e-12 & 257 \\
        Rosso & 2.80e-07 & 6.03e-07 & 215 \\
        Blu & 5.02e-06 & 9.04e-06 & 180 \\
        Verde & 1.43e-05 & 2.29e-05 & 161 \\
        Bianco & 4.13e-09 & 1.37e-08 & 331 \\
        Infrarosso & 2.45e-12 & 8.96e-12 & 365 \\
        Fotodiodo & 1.45e-04 & 7.93e-05 & 55 \\
        \bottomrule
    \end{tabular}
    \caption{Valori di $I_0$ per i diversi LED e per il fotodiodo}
\end{table}

\subsection{Plot}
Di seguito è riportato il grafico dei dati sperimentali con la curva di fit.

\subsection{Caratterizzazione elettrica}
\begin{figure}[H]
    \centering
    \includegraphics[width=0.55\textwidth]{Plot LED Giallo.png}
    \label{grafico_plot_giallo}
\end{figure}
\begin{figure}[H]
    \centering
    \includegraphics[width=0.55\textwidth]{Plot LED Rosso.png}
    \label{grafico_plot_rosso}
\end{figure}
\begin{figure}[H]
    \centering
    \includegraphics[width=0.55\textwidth]{Plot LED Blu.png}
    \label{grafico_plot_blu}
\end{figure}
\begin{figure}[H]
    \centering
    \includegraphics[width=0.55\textwidth]{Plot LED Verde.png}
    \label{grafico_plot_verde}
\end{figure}
\begin{figure}[H]
    \centering
    \includegraphics[width=0.55\textwidth]{Plot LED Bianco.png}
    \label{grafico_plot_bianco}
\end{figure}
\begin{figure}[H]
    \centering
    \includegraphics[width=0.55\textwidth]{Plot LED Infrarosso.png}
    \label{grafico_plot_infrarosso}
\end{figure}
\begin{figure}[H]
    \centering
    \includegraphics[width=0.55\textwidth]{Plot Fotodiodo.png}
    \label{grafico_plot_fotodiodo}
\end{figure}

% -------------------------------------------------------------------
\section{Conclusioni}
\textbf{Esiti fisici:}
Nella caratterizzazione elettrica si può osservare che i dati sperimentali seguono una curva esponenziale.
\\
\textbf{Commenti:}

\end{document}