\documentclass[a4paper,12pt]{article}

% --- Pacchetti Fondamentali ---
\usepackage[utf8]{inputenc} % Codifica caratteri
\usepackage[T1]{fontenc}    % Codifica font
\usepackage[italian]{babel} % Lingua italiana
\usepackage{geometry}       % Gestione margini
\usepackage{parskip}        % Gestione spazi tra paragrafi
\usepackage{hyperref}
\geometry{a4paper, margin=2.5cm}

% --- Pacchetti Matematica e Fisica ---
\usepackage{amsmath, amssymb} % Simboli matematici
\usepackage{siunitx}          % Gestione unità di misura e numeri (FONDAMENTALE)
\sisetup{
    output-decimal-marker = {.}, % Punto come separatore decimale
    separate-uncertainty = true, % Scrive errori come (Valore +/- Errore)
    per-mode = power,            % Usa / per le unità (m/s)
    inter-unit-product = \ensuremath{\cdot} % Per scrivere le unità con il punto (N·m)
}

% --- Pacchetti Immagini e Tabelle ---
\usepackage{graphicx} % Per inserire PNG/JPG
\usepackage{float}    % Per forzare la posizione delle immagini (H)
\usepackage{booktabs} % Per tabelle professionali (\toprule, \midrule)
\usepackage{caption}  % Per personalizzare le didascalie
\usepackage{subcaption}
\usepackage[font=small, labelfont=bf]{caption}
\usepackage[font=small, labelfont=bf]{subcaption}

% --- Dati Intestazione ---
\title{\textbf{Caratterizzazione elettrica e ottica dei LED}}
\author{Filippo Audisio, Cataldo Insalaco, Telemaco Pezzoni}
\date{\today}

\begin{document}

\maketitle

% -------------------------------------------------------------------
\section{Obiettivo dell'esperienza}
    L'obiettivo dell'esperienza è studiare la caratterizzazione elettrica e ottica di alcuni LED di colori diversi, nello specifico:
    \begin{itemize}
        \item Nella caratterizzazione elettrica verificare la legge esponenziale $I = I_0(e^\frac{qV}{nkT}-1)$ e calcolare il valore di $I_0$.
        \item Nella caratterizzazione ottica ricavare la lunghezza d'onda del massimo della luce emessa dai LED.
    \end{itemize}

% -------------------------------------------------------------------

\section{Materiali e Metodi}
    % T: idem per la procedura, per risparmiare spazio eventualmente si può unire la strumentazione in un solo blocco
    \subsection{Strumentazione}
        \subsubsection{Caratterizzazione elettrica}    
            \begin{itemize}
                \item LED colore bianco, blu, giallo, rosso, verde e infrarosso
                \item Fotodiodo
                \item Due multimetri
                \item Resistenza da $511 \Omega$
                \item Cavi a banana
                \item Alimentatore in continua
            \end{itemize}

        \subsubsection{Caratterizzazione ottica}
            \begin{itemize}
                \item LED colore bianco, blu, giallo, rosso, verde e infrarosso
                \item Fotodiodo
                \item Due multimetri
                \item Cavi a banana
                \item Alimentatore in continua
                \item Reticolo (1000 linee/mm)
                \item Scatola condizionamento segnale
                \item Binario e braccio rotante
                \item Tre supporti
                \item Due lenti
                \item Foglio con scala angolare
            \end{itemize}

\subsection{Procedura sperimentale}
    \subsubsection{Caratterizzazione elettrica}
    Collegare in serie un multimetro, usato come amperometro, il LED e la resistenza all'alimentatore in continua. 
    Collegare in parallelo al LED l'altro multimetro, utilizzandolo come voltmetro.
    Variando la corrente nell'alimentatore alimentare il LED registrando le coppie di valori tensione-corrente lette sui multimetri.
    Prendere nota dei valori di corrente per i quali si ha l'apparire di emissione luminosa da parte del LED.

    \subsubsection{Caratterizzazione ottica}
    Inserire il LED e il reticolo sui supporti ottici presenti sul binario, mentre il fotodiodo sul supporto che si trova sul braccio rotante, a una distanza di 39 cm dal reticolo.
    Collegare in serie un multimetro, utilizzato come amperometro, e il LED all'alimentatore in continua
    Collegare il fotodiodo all'altro multimetro, utilizzato come voltmetro, tramite la scatola di condizionameto del segnale, che permette di avere una risposta lineare del fotodiodo.
    Allineare il braccio rotante e il foglio con la scala angolare cercando prima l'ordine zero: si deve osservare un massimo di risposta in tensione del fotodiodo quando LED, reticolo e fotodiodo sono allineati.
    Successivamente verificare che ruotando l'asta l'estremità del braccio rotante continui a scorrere smepre sulla linea dei 39 cm presente sul foglio con scala angolare.
    Far scorrere l'asta rotante e prendere nota dei valori in tensione a diversi angoli.
    Per convertire l'angolo $\theta$ in lunghezza d'onda $\lambda$ bisogna usare la formula dell'interferenza: $\sin(\theta) = \frac{m\lambda}{d}$, dove in questo caso $m=1$ perchè quello che verrà misurato è solo il massimo al primo ordine.
    Questo calcolo, però non è da fare perchè sul foglio con scala angolare sono già presenti le lunghezze d'onda che corrispondono a determinati angoli.
    Quindi prendere nota dei valori in tensione in corrispondenza dei valori di lunghezza d'onda letti sul foglio. 

% -------------------------------------------------------------------

\section{Analisi dei dati e grafici}
    \subsection{Tabelle risultati}
        \subsubsection{Valori di $I_0$ nella caratterizzazione elettrica}
            Usando come funzione di fit $I = I_0(e^\frac{qV}{nkT}-1)$ abbiamo ottenuto come valori di $I_0$:
            \begin{table}[H]
                \centering
                \begin{tabular}{
                    l 
                    S[scientific-notation = true, table-format=1.2e2]   % Formato per il valore
                    S[scientific-notation = true, table-format=1.2e2]   % Formato per sigma
                    S[table-format=3.2] % Formato per l'errore relativo
                }
                    \toprule
                    {LED} & {Valore $I_0$ [\unit{A}]} & {$\sigma$} & {Errore Relativo $I_0$ [\%]} \\
                    \midrule
                    Giallo & 3.28e-12 & 8.44e-12 & 257 \\
                    Rosso & 2.80e-07 & 6.03e-07 & 215 \\
                    Blu & 5.02e-06 & 9.04e-06 & 180 \\
                    Verde & 1.43e-05 & 2.29e-05 & 161 \\
                    Bianco & 4.13e-09 & 1.37e-08 & 331 \\
                    Infrarosso & 2.45e-12 & 8.96e-12 & 365 \\
                    Fotodiodo & 1.45e-04 & 7.94e-05 & 55 \\
                    \bottomrule
                \end{tabular}
                \caption{Valori di $I_0$ per i diversi LED e per il fotodiodo}
                \label{tab:I0_LED}
            \end{table}
        \subsubsection{Tabella dei massimi di emissione e di $E_{gap}$}
            I dati riportati in seguito sono l'intensità massima $V_{max}$ e la relativa lunghezza d'onda $\lambda_{max}$ per i LED di diverso colore.
            Inoltre sono riportati i valori per $V_{soglia}$ che rappresenta la tensione a cui abbiamo osservato una prima emissione luminosa da parte del LED, e di $E_{gap}$ è l'energia di picco di un fotone, calcolata a partire da $\lambda_{max}$: $E_{gap} = \frac{h c}{\lambda_{max}}$.

            \begin{table}[H]
                \centering
                \begin{tabular}{
                    l 
                    S[table-format=3.0(2)]   % Formato per lambda
                    c   % Formato per lambda atteso
                    S[table-format=1.3(3)]   % Formato per V_soglia
                    S[table-format=1.3(3)]   % Formato per E_gap
                }
                    \toprule
                    {LED} & {Valore $\lambda_{max}$ [\unit{nm}]} & {Valore atteso per $\lambda_{max}$ [\unit{nm}]} & {Valore $V_{soglia}$ [\unit{V}]} & {Valore $E_{gap}$ [\unit{eV}]} \\
                    \midrule
                    Giallo & 610(10) & {[575 - 595]} & 1.773(0.001) & 2.034(0.033) \\
                    Rosso & 650(10) & {[620 - 700]} & 1.503(0.001) & 1.909(0.029) \\
                    Blu & 470(10) & {[460 - 475]} & 2.323(0.001) & 2.640(0.056) \\
                    Verde & 520(10) & 525 & 2.291(0.001) & 2.386(0.046) \\
                    Bianco & 560(10) & {[540 - 580]} & 2.439(0.001) & 2.216(0.040) \\
                    Infrarosso & 850(10) & {[840 - 870]} &  &  \\
                    \bottomrule
                \end{tabular}
                \caption{Valori di $\lambda_{max}$ e $E_{gap}$ per i diversi LED}
                \label{tab:lambda_LED}
            \end{table}

    \subsection{Grafici sperimentali e curve di regressione}
        \subsubsection{Caratterizzazione elettrica}
            \begin{figure}[H]
                \centering
                \includegraphics[width=0.9\textwidth]{Caratterizzazione_elettrica.png}
                \caption{Intensità della corrente in funzione della tensione}
                \label{fig:grafico_elettrica}
            \end{figure}
        \subsubsection{Caratterizzazione ottica}
            \begin{figure}[H]
                \centering
                \includegraphics[width=0.9\textwidth]{Caratterizzazione_ottica.png}
                \caption{Tensione letta ai capi del fotodiodo in funzione della lunghezza d'onda}
                \label{fig:grafico_ottica}
            \end{figure}
        \subsubsection{Grafico $E_{gap}$ - $V_{soglia}$}
            \begin{figure}[H]
                \centering
                \includegraphics[width=0.9\textwidth]{Egap_Vsoglia.png}
                \caption{$V_{soglia}$ in funzione di $E_{gap}$}
                \label{fig:grafico_Egap_Vsoglia}
            \end{figure}

% ---------------------------------------------------------------------------------------

\section{Conclusioni}
Nella caratterizzazione elettrica si è verificato l'andamento esponenziale (\autoref{fig:grafico_elettrica}) della corrente che passa in un LED in funzione della tensione ai capi di quest'ultimo. 
Gli errori sulle $I_0$ sono grandi, ma questo potrebbe essere dovuto al fit esponenziale.

Nella caratterizzazione ottica si osserva che i LED giallo, rosso, verde, blu hanno un massimo di emissione per una lunghezza d'onda coerente con lo spettro elettromagnetico del visibile, mentre il LED infrarosso presenta un picco per una lunghezza d'onda maggiore. 
Le lunghezze d'onda trovate sono coerenti con quelle attese, ad eccezione del giallo, ma questo potrebbe essere dovuto a un al fatto che il LED emetteva nella regione del giallo-ambra, invece che del giallo puro.
Il LED bianco sembra invece avere due picchi: uno che coincide con il picco del blu e l'altro con quello del giallo.
Questo è dovuto al fatto che il LED bianco è costruito a partire da un LED blu coperto da uno strato di fosforo che emette luce nel giallo.

Infine nel grafico che mette a confronto l'energia di gap e la tensione di soglia si può osservare che si ha un andamento circa lineare: l'energia di soglia aumenta con il diminuire di $\lambda$.
Quello del LED bianco è un dato che si discosta dall'andamento lineare e potrebbe essere dovuto a come è costruito questo LED. 
Alla base del LED bianco c'è un LED blu, quindi per riuscire a emettere luce bisogna superare la $V_{soglia}$ del blu.
Infatti, nella \autoref{fig:grafico_Egap_Vsoglia}, si può osservare che la $V_{soglia}$ per il LED bianco è maggiore/circa uguale a quella per il led blu.
\end{document}