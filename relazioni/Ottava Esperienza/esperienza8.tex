\documentclass[a4paper,12pt]{article}

% --- Pacchetti Fondamentali ---
\usepackage[utf8]{inputenc} % Codifica caratteri
\usepackage[T1]{fontenc}    % Codifica font
\usepackage[italian]{babel} % Lingua italiana
\usepackage{geometry}       % Gestione margini
\usepackage{parskip}        % Gestione spazi tra paragrafi
\usepackage{hyperref}
\geometry{a4paper, margin=2.5cm}

% --- Pacchetti Matematica e Fisica ---
\usepackage{amsmath, amssymb} % Simboli matematici
\usepackage{siunitx}          % Gestione unità di misura e numeri (FONDAMENTALE)
\sisetup{
    output-decimal-marker = {.}, % Punto come separatore decimale
    separate-uncertainty = true, % Scrive errori come (Valore +/- Errore)
    per-mode = power,            % Usa / per le unità (m/s)
    inter-unit-product = \ensuremath{\cdot} % Per scrivere le unità con il punto (N·m)
}

% --- Pacchetti Immagini e Tabelle ---
\usepackage{graphicx} % Per inserire PNG/JPG
\usepackage{float}    % Per forzare la posizione delle immagini (H)
\usepackage{booktabs} % Per tabelle professionali (\toprule, \midrule)
\usepackage{caption}  % Per personalizzare le didascalie
\usepackage{subcaption}
\usepackage[font=small, labelfont=bf]{caption}
\usepackage[font=small, labelfont=bf]{subcaption}


% --- Dati Intestazione ---
\title{\textbf{Attività ottica}}
\author{Filippo Audisio, Cataldo Insalaco, Telemaco Pezzoni}
\date{\today}

\begin{document}

\maketitle

% -------------------------------------------------------------------
\section{Obiettivo dell'esperienza}
L'obiettivo dell'esperienza è studiare il fenomeno dell'attività ottica per luce di diverse lunghezze d'onda attraverso varie soluzioni acquose. In particolare si vuole:
\begin{itemize}
    \item Verificare la legge di Biot prima in funzione della lunghezza di propagazione nel mezzo $L$ (secondo $\alpha = K \cdot L$) e successivamente in funzione della massa disciolta $P$ (secondo la relazione derivata $\alpha = k \cdot P/S$), in quest'ultima parte determinare anche il potere rotatorio specifico $k$ per la sostanza utilizzata.
    \item Misurare il potere rotatorio specifico di varie soluzioni, verificandone la dipendenza dalla lunghezza d'onda della luce incidente.
    \item Osservare e studiare il fenomeno di mutarotazione in soluzione di glucosio.
    \item Osservare e studiare il fenomeno di inversione della soluzione di saccarosio.
\end{itemize}

% -------------------------------------------------------------------
\section{Materiali e Metodi}

\subsection{Dotazione sperimentale}
\begin{itemize}
    \item Polarimetro con LED di lunghezza d'onda $\lambda$ variabile tra $468 \unit{nm}$ (blu), $525 \unit{nm}$ (verde), $580 \unit{nm}$ (giallo), $630 \unit{nm}$ (rosso).
    \item Cilindro graduato per polarimetro.
    \item Becher e cilindro graduato con base.
    \item Saccarosio, fruttosio e glucosio in polvere.
    \item Acqua.
    \item Piastra riscaldante.
    \item Soluzione HCl al $25\%$.
    \item Materiali di consumo.
\end{itemize}

\subsection{Procedura sperimentale}
Prima di iniziare le misure è stata verificata la taratura del polarimetro osservando i minimi di intensità luminosa con il cilindro graduato vuoto, così da poter successivamente tenere in conto eventuali offset. In seguito si è ripetuta l'operazione con il cilindro riempito di sola acqua verificando che gli angoli per il minimo di intensità non cambiano. Infine è stata ricavata indirettamente la sezione del cilindro graduato misurando l'altezza raggiunta da $100 \unit{ml}$ di acqua all'interno del cilindro stesso, utilizzando la formula $S = V/L$. 
\subsubsection{Verifica della legge di Biot}
Per verificare la legge di Biot in funzione di $L$ è stata preparata nel becher una soluzione di saccarosio, sciogliendo in $\sim80 \unit{ml}$ di acqua $\sim 30 \unit{g}$ di saccarosio. La soluzione poi è stata versata in quantità crescenti nel cilindro graduato del polarimetro e per ogni altezza $L$ sono stati misurati gli angoli di minimo di intensità con luce verde.
In seguito, per verificare la legge in funzione della massa disciolta $P$, sono state preparate diverse soluzioni sciogliendo nella stessa quantità di acqua quantità crescenti di saccarosio e misurando per ognuna l'angolo di rotazione, sempre con luce verde.
Infine è stato calcolato il potere rotatorio specifico $k$ della sostanza.
\subsubsection{Misura del potere rotatorio specifico}
Sono state preparate diverse soluzioni di saccarosio, fruttosio e glucosio sciogliendo in $80 \unit{ml}$ di acqua $30 \unit{g}$ di soluto. Per ognuna delle soluzioni è stato misurato l'angolo di rotazione con luce di diverse lunghezze d'onda (blu, verde, giallo, rosso) e calcolato il potere rotatorio specifico $k$ verificandone la dispersione. Infine è stato calcolato il potere rotatorio specifico per $\lambda = 589 \unit{nm}$. %T: non sono sicuro se l'ultima frase vada bene qui non essendo puramente procedura sperimentale
\subsubsection{Studio della mutarotazione del glucosio}
È stata preparata una soluzione sciogliendo $30 \unit{g}$ di glucosio in $80 \unit{ml}$ di acqua riscaldata sulla piastra così da accelerare la reazione. Subito dopo la preparazione della soluzione sono stati misurati gli angoli di rotazione ogni minuto utilizzando luce verde, fino a quando l'angolo non si è stabilizzato.
Al raggiungimento dell'equilibrio sono stati misurati gli angoli di rotazione alle varie lunghezze d'onda.
\subsubsection{Studio dell'inversione del saccarosio}
Dopo aver sciolto $30 \unit{g}$ di saccarosio in acqua, sono stati aggiunti $2.5 \unit{ml}$ di HCl al $25\%$ per catalizzare l'inversione. Sono dunque stati misurati gli angoli di rotazione a intervalli di diversi minuti utilizzando luce verde; Per garantire il raggiungimento dell'equilibrio le ultime misurazioni sono state effettuate il giorno seguente. 
% -------------------------------------------------------------------
\section{Analisi dei dati e grafici}

\subsection{Tabelle risultati}
\begin{table}[H]
    \centering
    \begin{tabular}{
        l
        S[table-format=1.3]  
        S[table-format=1.3]  
        S[table-format=2.2]  
    }
        \toprule
        {Grandezza} & {Valore Misurato} & {Errore Assoluto} & {Errore Relativo [\%]} \\
        \midrule
        $offset$ [\unit{deg}] & 2 & 1 & 50.0 \\
        $S$ [\unit{cm^2}] & 5.882 & 0.035 & 0.59 \\
        \bottomrule
    \end{tabular}
    \caption{Misure preliminari}
    \label{tab:Misure preliminari}
\end{table}
\subsubsection{Verifica della legge di Biot}
Utilizzando le formule $\alpha = K \cdot L$ e $\alpha = k \cdot P/S$, sono stati calcolati rispettivamente i coefficienti $K=k \cdot c$ e $k$ tramite regressione lineare.
\begin{table}[H]
    \centering
    \begin{tabular}{
        l
        S[table-format=1.3]  
        S[table-format=1.3]  
        S[table-format=2.2]  
    }
        \toprule
        {Grandezza} & {Valore Misurato} & {$\sigma$} & {Errore Relativo [\%]} \\
        \midrule
        $K$ [\unit{deg.cm^-1}] & 2.53 & 0.15 & 5.76 \\
        $k$ [\unit{deg.cm^2.g^-1}] & 8.05 & 0.41 & 5.03 \\
        \bottomrule
    \end{tabular}
    \caption{Verifica della legge di Biot: parametri per saccarosio con luce verde}
    \label{tab:Legge di Biot}
\end{table}
\subsubsection{Misura del potere rotatorio specifico}
Dopo aver misurato i valori di $k=\alpha \cdot S /P$, usando la formula $k = A \cdot \lambda^2$ sono stati ricavati i valori di $A$ da cui si è calcolato il valore di $k$ a $\lambda = 589 \unit{nm}$ per le varie sostanze utilizzate.
\begin{table}[H]
    \centering
    \begin{tabular}{
        l
        S[table-format=-2.2]  
        S[table-format=-1.2]  
        S[table-format=1.2]  
        S[table-format=2.2]  
    }
        \toprule
        {Sostanza} & {Valore Tabulato $k$} & {Valore Misurato $k$} & {$\sigma$} & {Errore Relativo [\%]} \\
        \midrule
        Saccarosio & 6.65 & 6.41 & 0.12 & 1.81 \\
        Fruttosio & -9.12 & -9.19 & 0.26 & 2.87 \\
        Glucosio & 11.22 & 8.61 & 0.23 & 2.66 \\
        \bottomrule
    \end{tabular}
    \caption{Misura del potere rotatorio specifico a $\lambda = 589 \unit{nm}$, tutte le misure sono espresse in [\unit{deg.cm^2.g^-1}]}
    \label{tab:Potere rotatorio specifico}
\end{table}
\subsubsection{Studio della mutarotazione del glucosio}
Con i dati raccolti è stato eseguito un fit esponenziale con la formula tipica delle reazioni del primo ordine: $\alpha = (\alpha_0 - \alpha_{\infty}) \cdot \exp^{-ct} + \alpha_{\infty}$, dai parametri si sono ricavati i valori $k_0^{525}$ e $k_{\infty}^{525}$ per una soluzione di glucosio in luce verde.
Inoltre dai dati presi all'equilibrio alle varie lunghezze d'onda si è calcolato $k_{\infty}^{589}$ come nella sezione precedente.
\begin{table}[H]
    \centering
    \begin{tabular}{
        l
        S[table-format=2.3]  
        S[table-format=2.3]
        S[table-format=1.3]  
        S[table-format=2.2]  
    }
        \toprule
        {Grandezza} & {Valore Tabulato}& {Valore Misurato} & {$\sigma$} & {Errore Relativo [\%]} \\
        \midrule
        $k_0^{525}$ & 12.82 & 11.03 & 0.71 & 6.45 \\
        $k_{\infty}^{525}$ & 5.98 & 5.35 & 0.63 & 11.81 \\
        $k_{\infty}^{589}$ & 4.76 & 4.98 & 0.34 & 6.81 \\
        \bottomrule
    \end{tabular}
    \caption{Potere rotatorio specifico del glucosio, tutte le misure sono espresse in [\unit{deg.cm^2.g^-1}]}
    \label{tab:Potere rotatorio specifico mutarotazione}
\end{table}
\subsubsection{Studio dell'inversione del saccarosio}
Come nella sezione precedente i parametri della soluzione catalizzata sono stati ottenuti tramite fit esponenziale, invece per la soluzione di riferimento si è usato un fit costante.
\begin{table}[H]
    \centering
    \begin{tabular}{
        l
        S[table-format=1.2]  
        S[table-format=-1.2]  
        S[table-format=1.2] 
        S[table-format=1.2] 
        S[table-format=-1.2]
        S[table-format=1.2]  
    }
        \toprule
        {Sostanza} & {$k_0^{tab}$} & {$k_{\infty}^{tab}$} & {$k_0^{mis}$} & ${\sigma_0}$ & {$k_{\infty}^{mis}$} & $\sigma_{\infty}$ \\
        \midrule
        Saccarosio + HCl & 8.37 & -2.44 & 8.49 & 0.67 & -2.09 & 0.52 \\
        Saccarosio (Rif.) & 8.37 & 8.37 & 7.49 & 0.35 & 7.49 & 0.35 \\
        \bottomrule
    \end{tabular}
    \caption{Misura del potere rotatorio specifico per l'inversione del saccarosio in luce verde $\lambda = 525 \unit{nm}$, tutte le misure sono espresse in [\unit{deg.cm^2.g^-1}]}
    \label{tab:Potere rotatorio specifico zucchero invertito}
\end{table}

\subsection{Grafici sperimentali e curve di regressione}

\subsubsection{Verifica della legge di Biot}
\begin{figure}[H]
     \centering
     \begin{subfigure}{0.48\textwidth}
         \centering
         \includegraphics[width=\linewidth]{neoneofitL.png}
         \caption{Angolo in funzione della lunghezza $L$}
         \label{fig:1a}
     \end{subfigure}
     \begin{subfigure}{0.48\textwidth}
         \centering
         \includegraphics[width=\linewidth]{neoneofitP.png}
         \caption{Angolo in funzione della massa disciolta $P$}
         \label{fig:1b}
     \end{subfigure}
     \caption{Verifica della legge di Biot}
     \label{fig:1}
\end{figure}

\subsubsection{Misura del potere rotatorio specifico}
\begin{figure}[H]
     \centering
     \includegraphics[width=0.9\linewidth]{Grafico potere rotatorio specifico.png}
     \caption{Potere rotatorio specifico in funzione della lunghezza d'onda}
     \label{fig:2}
\end{figure}

\subsubsection{Studio della mutarotazione del glucosio}
\begin{figure}[H]
    \centering
    \includegraphics[width=0.9\linewidth]{Fit_6_mutarotazione glucosio.png}
    \caption{Angolo di rotazione in funzione del tempo per la mutarotazione del glucosio}
    \label{fig:3}
\end{figure}

\subsubsection{Studio dell'inversione del saccarosio}
\begin{figure}[H]
    \centering
    \includegraphics[width=0.9\linewidth]{Grafico inversione HCl e Rif.png}
    \caption{Angolo di rotazione in funzione del tempo per l'inversione del saccarosio}
    \label{fig:4}
\end{figure}

% -------------------------------------------------------------------

\section{Conclusioni}
Tutte le misure degli angoli riportate tengono già conto dell'offset del polarimetro misurato all'inizio; inoltre si è scelto di ricavare la sezione del cilindro graduato indirettamente, piuttosto che misurare direttamente il diametro, così da minimizzare l'errore associato a tale misura.

Nella prima parte dell'esperienza si è verificata la proporzionalità diretta tra angolo di rotazione $\alpha$ e lunghezza di propagazione nel mezzo $L$ (\autoref{sub@fig:1a}), misurando anche il potere rotatorio
$K=k \cdot c$ relativo alla concentrazione utilizzata. Per ottenere una buona misura del potere rotatorio specifico $k$, invece che variare la concentrazione, si è misurato $\alpha$ in funzione
della massa disciolta $P$; il valore ottenuto per saccarosio a $\lambda = 525 \unit{nm}$ risulta coerente con quello misurato nella parte successiva (\autoref{fig:2}) ed in accordo con il valore tabulato (\autoref{tab:Potere rotatorio specifico zucchero invertito}). 

Nella seconda parte la dipendenza del potere rotatorio specifico dalla lunghezza d'onda della luce incidente è stata verificata per tutte le sostanze utilizzate (\autoref{fig:2}). I valori trovati per il saccarosio e il fruttosio sono in buon accordo con i valori tabulati, mentre 
la discrepanza che si riscontra con il glucosio è probabilmente dovuta al fenomeno di mutarotazione ed al tempo trascorso tra le misure (\autoref{tab:Potere rotatorio specifico}). Infatti si è utilizzato $\alpha$-D-glucosio puro
($k$ = 11.2 \unit{deg.cm^2.g^-1}), che in soluzione tende a trasformarsi in una miscela di $\alpha$ e $\beta$-D-glucosio ($k$ = 5.23 \unit{deg.cm^2.g^-1}); il valore misurato è pertanto
uno stato intermedio tra i due.

Nella terza parte si verifica che il glucosio in soluzione acquosa subisce il fenomeno di mutarotazione, in cui l'$\alpha$-D-glucosio iniziale si converte in una soluzione di equilibrio
36[\%] $\alpha$-D-glucosio e 64[\%] $\beta$-D-glucosio, i valori dell'angolo di rotazione variano seguendo una cinetica esponenziale, trattandosi di una reazione del primo ordine (\autoref{fig:3}). 
I valori misurati del potere rotatorio specifico all'equilibrio risultano compatibili con quelli tabulati (debitamente corretti per glucosio monoidrato $k_{{mono}} = k_{tab} \cdot 180/198$ e luce verde $k^{525}=k^{589} \cdot (589/525)^2$) (\autoref{tab:Potere rotatorio specifico mutarotazione}).
Ad eccezione del valore di $k_0^{525}$, probabilmente influenzato dal tempo trascorso tra la preparazione della soluzione e l'inizio delle misure.

Nell'ultima parte si osserva il fenomeno di inversione del saccarosio, in cui la molecola di saccarosio viene scissa in glucosio e fruttosio. Per la soluzione con HCl si osserva 
una variazione dell'angolo di rotazione che segue una cinetica esponenziale (\autoref{fig:4}), mentre la soluzione di riferimento senza HCl non mostra variazioni significative.
Si osserva che lo zucchero invertito è un mezzo levogiro, a differenza della soluzione iniziale destrogira, inoltre i valori misurati del potere rotatorio specifico iniziale e finale 
sono in accordo con i valori tabulati entro $3\sigma$ (\autoref{tab:Potere rotatorio specifico zucchero invertito}). La soluzione di riferimento mostra invece un valore di $k$ leggermente inferiore a quello tabulato,
ciò probabilmente è dovuto al ridotto numero di misure effettuate. %T: oppure una parte dello zucchero si è invertito spontaneamente?

\end{document}