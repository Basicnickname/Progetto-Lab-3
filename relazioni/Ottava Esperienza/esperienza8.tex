\documentclass[a4paper,12pt]{article}

% --- Pacchetti Fondamentali ---
\usepackage[utf8]{inputenc} % Codifica caratteri
\usepackage[T1]{fontenc}    % Codifica font
\usepackage[italian]{babel} % Lingua italiana
\usepackage{geometry}       % Gestione margini
\geometry{a4paper, margin=2.5cm}

% --- Pacchetti Matematica e Fisica ---
\usepackage{amsmath, amssymb} % Simboli matematici
\usepackage{siunitx}          % Gestione unità di misura e numeri (FONDAMENTALE)
\sisetup{
    output-decimal-marker = {.}, % Punto come separatore decimale
    separate-uncertainty = true, % Scrive errori come (Valore +/- Errore)
    per-mode = symbol            % Usa / per le unità (m/s)
}

% --- Pacchetti Immagini e Tabelle ---
\usepackage{graphicx} % Per inserire PNG/JPG
\usepackage{float}    % Per forzare la posizione delle immagini (H)
\usepackage{booktabs} % Per tabelle professionali (\toprule, \midrule)
\usepackage{caption}  % Per personalizzare le didascalie

% --- Dati Intestazione ---
\title{\textbf{Esperienza sull'attività ottica}}
\author{Filippo Audisio, Cataldo Insalaco, Telemaco Pezzoni}
\date{\today}

\begin{document}

\maketitle

% -------------------------------------------------------------------
\section{Obiettivo dell'esperienza}
L'obiettivo dell'esperienza è studiare il fenomeno dell'attività ottica per luce di diverse lunghezze d'onda attraverso varie soluzioni acquose. In particolare si vuole:
\begin{itemize}
    \item Verificare la legge di Biot: $\alpha = k c L$. 
    \item Misurare il potere rotatorio specifico di varie soluzioni, verificandone la dipendenza dalla lunghezza d'onda della luce incidente.
    \item Osservare e studiare il fenomeno di mutarotazione in soluzione di glucosio.
    \item Osservare e studiare il fenomeno di inversione della soluzione di saccarosio.
\end{itemize}

% -------------------------------------------------------------------
\section{Materiali e Metodi}

\subsection{Dotazione sperimentale}
\begin{itemize}
    \item Polarimetro con LED di lunghezza d'onda $\lambda$ variabile tra $468 \unit{nm}$ (blu), $525 \unit{nm}$ (verde), $580 \unit{nm}$ (giallo), $630 \unit{nm}$ (rosso).
    \item Cilindro graduato per polarimetro.
    \item Becher e cilindro graduato con base.
    \item Saccarosio, fruttosio e glucosio in polvere.
    \item Acqua.
    \item Piastra riscaldante.
    \item Soluzione HCl al $25\%$.
    \item Materiali di consumo.
\end{itemize}

\subsection{Procedura sperimentale}
\subsubsection{Verifica della legge di Biot}
\subsubsection{Misura del potere rotatorio specifico}
\subsubsection{Studio della mutarotazione del glucosio}
\subsubsection{Studio dell'inversione del saccarosio}
% -------------------------------------------------------------------
\section{Dati sperimentali e Analisi}

\subsection{Grafici dati sperimentali}
\begin{figure}[H]
\end{figure}

\subsection{Tabelle risultati}
\begin{table}[H]
    \centering
    \begin{tabular}{
        l 
        S[scientific-notation = true, table-format=1.2e2]   % Formato per il valore
        S[scientific-notation = true, table-format=1.2e2]   % Formato per l'errore
        S[table-format=3.2] % Formato per l'errore relativo
    }
        \toprule
        \midrule
        \bottomrule
    \end{tabular}
    \caption{}
    \label{tab}
\end{table}

\subsection{Plot}
Di seguito sono riportati i grafici di confronto tra i dati sperimentali e le curve teoriche.

\begin{figure}[H]
    \centering
\end{figure}


% -------------------------------------------------------------------

\section{Conclusioni}
\end{document}