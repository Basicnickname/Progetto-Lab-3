\documentclass[a4paper,12pt]{article}

% --- Pacchetti Fondamentali ---
\usepackage[utf8]{inputenc} % Codifica caratteri
\usepackage[T1]{fontenc}    % Codifica font
\usepackage[italian]{babel} % Lingua italiana
\usepackage{geometry}       % Gestione margini
\geometry{a4paper, margin=2.5cm}

% --- Pacchetti Matematica e Fisica ---
\usepackage{amsmath, amssymb} % Simboli matematici
\usepackage{siunitx}          % Gestione unità di misura e numeri (FONDAMENTALE)
\sisetup{
    output-decimal-marker = {.}, % Punto come separatore decimale
    separate-uncertainty = true, % Scrive errori come (Valore +/- Errore)
    per-mode = symbol            % Usa / per le unità (m/s)
}

% --- Pacchetti Immagini e Tabelle ---
\usepackage{graphicx} % Per inserire PNG/JPG
\usepackage{float}    % Per forzare la posizione delle immagini (H)
\usepackage{booktabs} % Per tabelle professionali (\toprule, \midrule)
\usepackage{caption}  % Per personalizzare le didascalie
\usepackage{subcaption}
\usepackage[font=small, labelfont=bf]{caption}

% --- Dati Intestazione ---
\title{\textbf{Attività ottica}}
\author{Filippo Audisio, Cataldo Insalaco, Telemaco Pezzoni}
\date{\today}

\begin{document}

\maketitle

% -------------------------------------------------------------------
\section{Obiettivo dell'esperienza}
L'obiettivo dell'esperienza è studiare il fenomeno dell'attività ottica per luce di diverse lunghezze d'onda attraverso varie soluzioni acquose. In particolare si vuole:
\begin{itemize}
    \item Verificare la legge di Biot prima in funzione della lunghezza di propagazione nel mezzo $L$ (secondo $\alpha = k c L$) e successivamente in funzione della massa disciolta $P$ (secondo la relazione derivata $\alpha = k P/S$), in quest'ultima parte determinare anche il potere rotatorio specifico $k$ per la sostanza utilizzata.
    \item Misurare il potere rotatorio specifico di varie soluzioni, verificandone la dipendenza dalla lunghezza d'onda della luce incidente.
    \item Osservare e studiare il fenomeno di mutarotazione in soluzione di glucosio.
    \item Osservare e studiare il fenomeno di inversione della soluzione di saccarosio.
\end{itemize}

% -------------------------------------------------------------------
\section{Materiali e Metodi}

\subsection{Dotazione sperimentale}
\begin{itemize}
    \item Polarimetro con LED di lunghezza d'onda $\lambda$ variabile tra $468 \unit{nm}$ (blu), $525 \unit{nm}$ (verde), $580 \unit{nm}$ (giallo), $630 \unit{nm}$ (rosso).
    \item Cilindro graduato per polarimetro.
    \item Becher e cilindro graduato con base.
    \item Saccarosio, fruttosio e glucosio in polvere.
    \item Acqua.
    \item Piastra riscaldante.
    \item Soluzione HCl al $25\%$.
    \item Materiali di consumo.
\end{itemize}

\subsection{Procedura sperimentale}
Prima di iniziare le misure è stata verificata la taratura del polarimetro osservando i minimi di intensità luminosa con il cilindro graduato vuoto, così da poter successivamente tenere in conto eventuali offset. In seguito si è ripetuta l'operazione con il cilindro riempito di sola acqua verificando che gli angoli per il minimo di intensità non cambiano. Infine è stata ricavata indirettamente la sezione del cilindro graduato misurando l'altezza raggiunta da $100 \unit{ml}$ di acqua all'interno del cilindro stesso, utilizzando la formula $S = V/L$ così da minimizzare l'errore di misura. 
\subsubsection{Verifica della legge di Biot}
Per verificare la legge di Biot in funzione di $L$ è stata preparata nel becher una soluzione di saccarosio, sciogliendo in $\sim80 \unit{ml}$ di acqua $\sim 30 \unit{g}$ di saccarosio. La soluzione poi è stata versata in quantità crescenti nel cilindro graduato del polarimetro e per ogni altezza $L$ sono stati misurati gli angoli di minimo di intensità con luce verde.
In seguito, per verificare la legge in funzione della massa disciolta $P$, sono state preparate diverse soluzioni sciogliendo nella stessa quantità di acqua quantità crescenti di saccarosio e misurando per ognuna l'angolo di rotazione, sempre con luce verde.
Infine è stato calcolato il potere rotatorio specifico $k$ della sostanza.
\subsubsection{Misura del potere rotatorio specifico}
Sono state preparate diverse soluzioni di saccarosio, fruttosio e glucosio sciogliendo in $80 \unit{ml}$ di acqua $30 \unit{g}$ di soluto. Per ognuna delle soluzioni è stato misurato l'angolo di rotazione con luce di diverse lunghezze d'onda (blu, verde, giallo, rosso) e calcolato il potere rotatorio specifico $k$ verificandone la dispersione. Infine è stato calcolato il potere rotatorio specifico per $\lambda = 589 \unit{nm}$. %T: non sono sicuro se l'ultima frase vada bene qui non essendo puramente procedura sperimentale
\subsubsection{Studio della mutarotazione del glucosio}
È stata preparata una soluzione sciogliendo $30 \unit{g}$ di glucosio in $80 \unit{ml}$ di acqua riscaldata sulla piastra così da accelerare la reazione. Subito dopo la preparazione della soluzione sono stati misurati gli angoli di rotazione ogni minuto utilizzando luce verde, fino a quando l'angolo non si è stabilizzato.
\subsubsection{Studio dell'inversione del saccarosio}
Dopo aver sciolto $30 \unit{g}$ di saccarosio in acqua, sono stati aggiunti $2.5 \unit{ml}$ di HCl al $25\%$ per catalizzare l'inversione. Sono dunque stati misurati gli angoli di rotazione a intervalli di diversi minuti utilizzando luce verde; Per garantire il raggiungimento dell'equilibrio le ultime misurazioni sono state effettuate il giorno seguente. 
% -------------------------------------------------------------------
\section{Dati sperimentali e Analisi}

\subsection{Grafici dati sperimentali}
\subsubsection{Verifica della legge di Biot}
\begin{figure}[H]
\begin{subfigure}[b]{\textwidth}
        \centering
        \includegraphics[width=\textwidth]{Dati_1_Funzione L.png}
        \caption{Angolo di rotazione in funzione della lunghezza di propagazione L}
        \label{fig:Biot L}
    \end{subfigure}
    \hfill
    \begin{subfigure}[b]{\textwidth}
        \centering
        \includegraphics[width=\textwidth]{Dati_2_Funzione P.png}
        \caption{Angolo di rotazione in funzione della massa disciolta P}
        \label{Biot P}
\end{subfigure}
\end{figure}
\subsubsection{Misura del potere rotatorio specifico}
\subsubsection{Studio della mutarotazione del glucosio}
\subsubsection{Studio dell'inversione del saccarosio}
\begin{figure}[H]
\end{figure}

\subsection{Tabelle risultati}
\begin{table}[H]
    \centering
    \begin{tabular}{
        l 
        S[scientific-notation = true, table-format=1.2e2]   % Formato per il valore
        S[scientific-notation = true, table-format=1.2e2]   % Formato per l'errore
        S[table-format=3.2] % Formato per l'errore relativo
    }
        \toprule
        \midrule
        \bottomrule
    \end{tabular}
    \caption{}
    \label{tab}
\end{table}

\subsection{Plot}
Di seguito sono riportati i grafici di confronto tra i dati sperimentali e le curve teoriche.

\begin{figure}[H]
    \centering
\end{figure}


% -------------------------------------------------------------------

\section{Conclusioni}
\end{document}