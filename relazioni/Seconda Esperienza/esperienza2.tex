\documentclass[a4paper,12pt]{article}

% --- Pacchetti Fondamentali ---
\usepackage[utf8]{inputenc} % Codifica caratteri
\usepackage[T1]{fontenc}    % Codifica font
\usepackage[italian]{babel} % Lingua italiana
\usepackage{geometry}       % Gestione margini
\usepackage{parskip}        % Gestione spazi tra paragrafi
\usepackage{hyperref}
\geometry{a4paper, margin=2.5cm}

% --- Pacchetti Matematica e Fisica ---
\usepackage{amsmath, amssymb} % Simboli matematici
\usepackage{siunitx}          % Gestione unità di misura e numeri (FONDAMENTALE)
\sisetup{
    output-decimal-marker = {.}, % Punto come separatore decimale
    separate-uncertainty = true, % Scrive errori come (Valore +/- Errore)
    per-mode = power,            % Usa / per le unità (m/s)
    inter-unit-product = \ensuremath{\cdot} % Per scrivere le unità con il punto (N·m)
}

% --- Pacchetti Immagini e Tabelle ---
\usepackage{graphicx} % Per inserire PNG/JPG
\usepackage{float}    % Per forzare la posizione delle immagini (H)
\usepackage{booktabs} % Per tabelle professionali (\toprule, \midrule)
\usepackage{caption}  % Per personalizzare le didascalie
\usepackage{subcaption}
\usepackage[font=small, labelfont=bf]{caption}
\usepackage[font=small, labelfont=bf]{subcaption}

% --- Dati Intestazione ---
\title{\textbf{Lettura da un fotodiodo polarizzato inversamente}}
\author{Filippo Audisio, Cataldo Insalaco, Telemaco Pezzoni}
\date{\today}

\begin{document}

\maketitle

% -------------------------------------------------------------------
\section{Obiettivo dell'esperienza}
    Il presente lavoro di laboratorio si è posto l'obiettivo di investigare sulle risposte di un fotodiodo (PD) in due regimi operativi fondamentali:
    \begin{itemize}
        \item Modalità fotoconduttiva (polarizzazione inversa, $V_g < 0$): la tensione di uscita ($V_R$) è direttamente proporzionale alla fotocorrente ($I_{ph}$). Determinare la costante di accoppiamento ottico $\beta$, che lega $I_{ph}$ alla corrente del LED ($I_{LED}$).
    \end{itemize}
    \begin{itemize}
        \item Modalità fotovoltaica (assenza di polarizzazione, $V_g = 0$). Ricavare i parametri intrinseci del diodo, la corrente di saturazione inversa ($I_0$) e il fattore di idealità ($n$).
    \end{itemize}

% -------------------------------------------------------------------

\section{Materiali e Metodi}
    \subsection{Strumentazione}
   \begin{itemize}
    \item  Fotodiodo (PD) 
    \item  LED
    \item  Resistenze di carico ($R$): due resistenze da \qty{10}{\kilo\ohm} e \qty{20}{\kilo\ohm}
    \item  Resistenza di protezione ($R_{LED}$) 
    \item  Due generatori d tensione 
    \item  Multimetri digitali  
 \end{itemize}

\subsection{Procedura sperimentale}
   Lo schema sperimentale è costituito da due circuiti indipendenti
\begin{enumerate}
    \item \textbf{Circuito 1}: Composto dal Generatore 1, un amperometro, il LED e la resistenza $R_{LED} = \qty{500}{\ohm}$. Quest'ultima facilita la regolazione della corrente tramite l'alimentatore e limita la corrente massima per preservare l'integrità del LED. L'intensità della radiazione emessa è regolata modulando la corrente di alimentazione $I_{LED}$.
    \item \textbf{Circuito 2}: Composto dal Generatore 2 (per la gestione della tensione di polarizzazione $V_g$), il fotodiodo e la resistenza di carico $R$. Un voltmetro è posto ai capi della resistenza per misurare la tensione di uscita $V_R$.
\end{enumerate}

È fondamentale che il LED e il fotodiodo siano allineati con precisione l'uno di fronte all'altro per garantire che la fotocorrente generata sia massima e che la costante di accoppiamento $\beta$ rimanga stabile durante le misure.
Sarebbe ottimo cercare di ridurre al minimo il disturbo della luce ambientale a cui il fotodiodo è sensibile e per  farlo si potrebbe usare un tubicino in cui inserire il fotodiodo e il led.
Il fotodiodo è estremamente sensibile. Senza un tubicino, il voltmetro registrerebbe una tensione $V_R$ influenzata anche dalle luci della stanza o dai monitor, falsando la misura della fotocorrente $I_{ph}$ generata esclusivamente dal LED. Il tubicino garantisce che le condizioni di "buio" siano quanto più reali possibile.
Per ogni set di misure, è stata impostata tramite il Generatore 2 una tensione di polarizzazione $V_g$. Le misure sono state effettuate partendo da $V_g = \qty{0}{\volt}$ (modalità fotovoltaica) e procedendo con valori in modulo sempre crescenti (polarizzazione inversa, $V_g < 0$).
Per un valore di $V_g$ fissato, si è variata la corrente di alimentazione del LED ($I_{LED}$) agendo sul Generatore 1, mantenendo sempre l'intensità sotto la soglia di sicurezza di \qty{20}{\milli\ampere}.
Per ogni incremento di $I_{LED}$, è stata registrata la tensione $V_R$ ai capi della resistenza di carico tramite il voltmetro digitale.
Per garantire un miglior dato sperimentale si è atteso un intervallo di alcuni secondi per assecondare i tempi di risposta della strumentazione e permettere segnale di stabilizzarsi.
Durante l'acquisizione, i dati sono stati riportati su un grafico $I_{LED}$ in funzione di $V_R$ per verificare immediatamente la coerenza con l'andamento teorico atteso.
Questo campionamento sistematico è stato ripetuto per due diverse resistenze di carico ($R = \qty{10}{\kilo\ohm}$ e $R = \qty{20}{\kilo\ohm}$), al fine di analizzare lo spostamento del punto di lavoro del circuito sulla retta di carico.
Inoltre usando due resistenze si ci può rendere conto che il modello fisico che si sta utilizzando sia  meno corretto, in quanto permette di dimostrare che i valori di $n$ e $I_0$ estratti dal fit sono proprietà reali del semiconduttore e non dipendono dal circuito di misura scelto.
Si può dunque verificare confrontando i risultati dei fit, in cui $\beta_1$, $\beta_2$, sono stati ricavati dai dati presi rispettivamente per  $R = 10\text{ k}\Omega$ e $V_g=0.3$ e $R = 20\text{ k}\Omega$ e $V_g=1$.
Mentre $I_{01}$, $n_1$,$I_{02}$ , $n_2$ sono stati ricavati dai  presi rispettivamente per  $R = 10\text{ k}\Omega$ e $R = 20\text{ k}\Omega$, entrambi con  $V_g=0$.


% -------------------------------------------------------------------

\section{Analisi dei dati e grafici}
    \subsection{Tabelle risultati}
         Utilizzando le seguenti formule calcoliamo rispettivamente:
\begin{itemize}
    \item Il fattore di idealità ($n$): annulliamo la polarizzazione,  Vg=0\unit{V}  ed utilizziamo la formula (con K costante di Boltzman e T temperatura ambientale(300K)):
\begin{equation}
    I_{LED} = \frac{1}{\beta} \left[ I_0 \left( e^{\frac{q V_R}{n k T}} - 1 \right) + \frac{V_R}{R} \right] \label{eq:fit_completo}
\end{equation}

    \item La corrente di saturazione inversa ($I_0$): annulliamo la polarizzazione,  Vg=0V  ed utilizziamo la formula precedente (sempre con K costante di Boltzman e T temperatura ambientale(300K)).

    \item La costante di accoppiamento ottico ($\beta$): polarizziamo con un valore grande e negativo di  $V_g$  e fittiamo utilizzando la relazione:
\begin{equation}
I_{LED} = \frac{V_R}{R \beta}
\end{equation}
\end{itemize}


\begin{table}[h]
\centering
\caption{Parametri di fit: valori, errori standard ed errori relativi.}
\label{tab:dati_fit}
\begin{tabular}{l S[table-format=1.2e-2] S[table-format=1.2e-2] S[table-format=3.2]}
\toprule
{Parametro} & {Valore} & {$\sigma$} & {Err. Rel. [\%]} \\
\midrule
$\beta_1$    & 2.59e-03 & 0.06e-03 & 2.38 \\
$\beta_2$    & 2.60e-03 & 0.07e-03 & 2.72 \\
$I_{01}[A]$    & 1.09e-09 & 1.90e-09 & 173 \\
$n_1$        & 1.24     & 0.23 & 18.2 \\
$I_{02}[A]$    & 2.58e-09 & 4.37e-09 & 169 \\
$n_2$        & 1.29     & 0.24 & 18.3 \\
\bottomrule
\end{tabular}
\end{table}


    \subsection{Grafici sperimentali e curve di regressione}

\begin{figure}[H]
  \centering
  \includegraphics[width=\textwidth]{Fit unito R1.png}
  \caption{Andamento della tensione $V_R$ in funzione di $I_{LED}$ per $R = 10\text{ k}\Omega$ e vari $V_g$}
  \label{fig:risultati_10k}
\end{figure}

\begin{figure}[H]
  \centering
  \includegraphics[width=\textwidth]{Fit unito R2.png}
  \caption{Andamento della tensione $V_R$ in funzione di $I_{LED}$ per $R = 20\text{ k}\Omega$ e vari $V_g$}
  \label{fig:risultati_20k}
\end{figure}


% ---------------------------------------------------------------------------------------

\section{Conclusioni}
I parametri $\beta$ sono più solidi e precisi ($err_{rel} \approx 2-3\%$), mentre i parametri $I_0$ e $n$ presentano incertezze maggiori.
 Il valore $n\simeq1.2$ indica la presenza di fenomeni di ricombinazione non radiativa nella regione di svuotamento.
I risultati confermano l'andamento esponenziale, ma presentano un errore molto elevato nei valori di $I_0$.
Questa elevata incertezza è probabilmente legata all influenza della luce ambientale.
Concludiamo che i risultati ottenuti confermano la validità del modello teorico.
Per studi futuri, si raccomanda l'adozione di una metodologia per isolare la risposta del LED dalle interferenze luminose esterne.





 
\end{document}
