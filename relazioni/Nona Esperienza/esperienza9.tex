\documentclass[a4paper,12pt]{article}

% --- Pacchetti Fondamentali ---
\usepackage[utf8]{inputenc} % Codifica caratteri
\usepackage[T1]{fontenc}    % Codifica font
\usepackage[italian]{babel} % Lingua italiana
\usepackage{geometry}       % Gestione margini
\usepackage{parskip}        % Gestione spazi tra paragrafi
\usepackage{hyperref}
\geometry{a4paper, margin=2.5cm}

% --- Pacchetti Matematica e Fisica ---
\usepackage{amsmath, amssymb} % Simboli matematici
\usepackage{siunitx}          % Gestione unità di misura e numeri (FONDAMENTALE)
\sisetup{
    output-decimal-marker = {.}, % Punto come separatore decimale
    separate-uncertainty = true, % Scrive errori come (Valore +/- Errore)
    per-mode = power,            % Usa / per le unità (m/s)
    inter-unit-product = \ensuremath{\cdot} % Per scrivere le unità con il punto (N·m)
}

% --- Pacchetti Immagini e Tabelle ---
\usepackage{graphicx} % Per inserire PNG/JPG
\usepackage{float}    % Per forzare la posizione delle immagini (H)
\usepackage{booktabs} % Per tabelle professionali (\toprule, \midrule)
\usepackage{caption}  % Per personalizzare le didascalie
\usepackage{subcaption}
\usepackage[font=small, labelfont=bf]{caption}
\usepackage[font=small, labelfont=bf]{subcaption}


% --- Dati Intestazione ---
\title{\textbf{Polarizzazione della luce e lamine di ritardo}}
\author{Filippo Audisio, Cataldo Insalaco, Telemaco Pezzoni}
\date{\today}

\begin{document}

\maketitle

% -------------------------------------------------------------------
\section{Obiettivo dell'esperienza}
L'obiettivo dell'esperienza è studiare il comportamento della luce polarizzata e delle lamine di ritardo, in particolare:
\begin{itemize}
    \item Verificare la legge di Malus $I(\theta)=I_0 \cos^2(\theta)$.
    \item Studiare l'utilizzo della lamina $\lambda/2$.
    \item Studiare l'utilizzo della lamina $\lambda/4$.
    \item Mostrare che la luce del LED non è polarizzata.
    \item Calcolare i parametri di Stokes per diversi stati di polarizzazione.
\end{itemize}

% -------------------------------------------------------------------
\section{Materiali e Metodi}

\subsection{Dotazione sperimentale}
\begin{itemize}
    \item LED e fotodiodo + scatola di polarizzazione.
    \item 2 lenti per la focalizzazione.
    \item 2 polarizzatori lineari.
    \item Lamine di ritardo $\lambda/2$ e $\lambda/4$.
    \item Supporti con ghiera rotante graduata.
    \item Barra di supporto.
    \item Generatore di tensione alternata.
    \item Oscilloscopio.
    \item Cavi coassiali e sdoppiatore.
\end{itemize}

\subsection{Procedura sperimentale}
Come configurazione iniziale sono stati posizionati il LED ed il fotodiodo alle estremità della barra di supporto.
Dal generatore di tensione alternata è stata sdoppiato il segnale di un'onda quadra a $\sim 6 \unit{V}$ e $\sim 1 \unit{kHz}$, il quale è stato mandato al LED per alimentarlo e ad un canale dell'oscilloscopio come trigger.
Il fotodiodo è stato polarizzato inversamente, grazie alla scatola di polarizzazione, e collegato all'altro canale dell'oscilloscopio così da poter misurare i valori di intensità trasmessa.
Sia al LED che al fotodiodo sono state applicate delle lenti per focalizzare il fascio luminoso, successivamente è stato montato un primo polarizzatore lineare sul LED ed un secondo sul supporto graduato posizionato fra LED e fotodiodo.
Infine usando il secondo polarizzatore come analizzatore si è trovato l'angolo di massima trasmissione così da misurare l'eventuale offset dovuto allo sfasamento tra i due polarizzatori. 
\subsubsection{Verifica della legge di Malus}
Dopo aver misurato il valore massimo di intensità trasmessa $V_0$, si è ruotato gradualmente l'analizzatore così da misurare $V$ in funzione di $\theta_{a}$.
\subsubsection{\texorpdfstring{Studio della lamina $\lambda/2$}{Studio della lamina lambda/2}}
Dopo aver inserito la lamina $\lambda/2$ fra i due polarizzatori e aver allineato il sistema così da misurare il massimo dell'intensità, ruotando la lamina di un angolo $\theta_{\lambda/2}$ sono stati misurati con l'analizzatore gli angoli $\theta_a^{max}$ e $\theta_a^{min}$ corrispondenti al massimo ed al minimo di trasmissione.
\subsubsection{\texorpdfstring{Studio della lamina $\lambda/4$}{Studio della lamina lambda/4}}
Dopo aver inserito la lamina $\lambda/4$ fra i due polarizzatori e rimosso momentaneamente l'analizzatore, si è innanzitutto verificato che ruotando la lamina l'intensità misurata dal fotodiodo non variasse. 
Successivamente è stato rimontato l'analizzatore e si è misurata l'intensità uscente in funzione dell'angolo di rotazione dell'analizzatore $\theta_a$ con lamina posizionata a $\theta_{\lambda/4} = 0^\circ, 90^\circ, 45^\circ, 70^\circ, 120^\circ$.
\subsubsection{Polarizzazione della luce del LED}
Togliendo tutti gli elementi ottici fra LED e fotodiodo (escluse le lenti di focalizzazione) si è misurata l'intensità rilevata dal fotodiodo $V_{LED}$, dopodiché è stato montato l'analizzatore e si è misurata l'intensità in funzione dell'angolo.
Dopodiché si è rimontata la lamina $\lambda/4$ tra LED e analizzatore e si è misurata l'intensità trasmessa con l'analizzatore posizionato a $45^\circ$ rispetto alla lamina; si è infine verificato che il risultato non cambiasse ruotando la lamina o l'analizzatore.
\subsubsection{Calcolo dei parametri di Stokes}
Per avere luce polarizzata linearmente si è rimontato il sistema nella configurazione iniziale con solamente polarizzatore lineare sul LED ed analizzatore, per luce polarizzata
circolarmente si è aggiunto tra i due elementi una lamina $\lambda/4$ a $45^\circ$ rispetto al primo polarizzatore, mentre per luce polarizzata ellitticamente si è orientata la lamina $\lambda/4$ a $70^\circ$ e poi 
$120^\circ$. Per ciascuna di queste configurazioni si è misurata l'intensità trasmessa con analizzatore posizionato a $0^\circ, 45^\circ, 90^\circ$, dopodiché si è inserita una seconda lamina $\lambda/4$ a $90^\circ$ e si è misurata l'intensità con analizzatore a $45^\circ$.
% -------------------------------------------------------------------
\section{Analisi dei dati e grafici}
\subsection{Tabelle risultati}
Tutti gli errori riportati corrispondono a deviazioni standard $\sigma$.
\subsubsection{Verifica della legge di Malus}
I parametri sono stati ricavati da un fit della legge di Malus: $I(\theta_{a}) = A \cdot \cos^2(\theta_{a} - \theta_{off}) + I^{fondo} $, normalizzando i valori di intensità misurati. 
\begin{table}[H]
    \centering
    \begin{tabular}{
        l
        S[table-format=-2(2)]
        S[table-format=-2.2(2)]  
        S[table-format=2.2]  
    }
        \toprule
        {Grandezza} & {Valore Teorico} & {Valore Ricavato} & {Errore Relativo [\%]} \\
        \midrule 
        A & 1 & 0.97(0.02) & 1.96 \\
        $\theta_{off}$ [\unit{deg}] & -10(2) & -9.2(0.6) & 6.39 \\
        $I^{fondo}$ [\unit{mV}] & 0 & 0.06(0.01) & 19.44 \\
        \bottomrule
    \end{tabular}
    \caption{Parametri della luce polarizzata linearmente.}
    \label{tab:Verifica Malus}
\end{table}

\subsubsection{\texorpdfstring{Studio della lamina $\lambda/2$}{Studio della lamina lambda/2}}
I parametri sono stati ottenuti da un fit della relazione: $\theta_{a} = m \cdot (\theta_{\lambda/2} - \theta_{off}') + K$, ripetuta per $\theta_{a}$ corrispondenti a massimi e minimi.
\begin{table}[H]
    \centering
    \begin{tabular}{
        l
        S[table-format=2]  
        S[table-format=2]  
        S[table-format=2]  
    }
        \toprule
        {Misura} & {$m$} & {$\theta_{off}'$ [\unit{deg}]} & {$K$ [\unit{deg}]} \\
        \midrule 
        Massimi & 2 & 20 & 0 \\
        Minimi & 2 & 20 & 90 \\
        \bottomrule
    \end{tabular}
    \caption{Relazione lineare tra angolo di rotazione della lamina e angolo dell'analizzatore.}
    \label{tab:Studio lamina lambda/2}
\end{table}

\subsubsection{\texorpdfstring{Studio della lamina $\lambda/4$}{Studio della lamina lambda/4}}
Per la luce polarizzata linearmente i dati sono stati analizzati con la legge di Malus, normalizzando i valori di intensità misurati.
\begin{table}[H]
    \centering
    \begin{tabular}{
        S[table-format=2]  
        S[table-format=1.2(2)]  
        S[table-format=1.2(2)]
    }
    \toprule
    {$\theta_{\lambda/4}$ [\unit{deg}]} & {$A$} & {$V_{norm}^{fondo}$ [\unit{mV}]}\\
    \midrule 
    0 & 0.93(0.05) & 0.08(0.04) \\
    90 & 0.94(0.01) & 0.07(0.01) \\
    \bottomrule
    \end{tabular}
\caption{Parametri della luce polarizzata linearmente ottenuti dal fit della legge di Malus.}
\label{tab:Studio lamina lambda/4 lineare}
\end {table}

Invece per i restanti casi è stata usata la funzione più generale: $I(\theta_a) = I_x \cdot \cos^2(\theta_a - \theta_1) + I_y \cdot \sin^2(\theta_a - \theta_1)$.
$\theta_1$ corrisponde all'orientazione di uno degli assi principali dell'ellisse di polarizzazione, si può ricavare $\theta_2 = \theta_1 + 90^\circ$.
\begin{table}[H]
    \centering
    \begin{tabular}{
    S[table-format=3]
    S[table-format=3(1)]
    S[table-format=3(1)]
    S[table-format=-3(1)]
    S[table-format=-3(1)]        
    }
    \toprule
    {$\theta_{\lambda/4}$ [$\unit{deg}$]} & {$I_x$ [$\unit{mV}$]} & {$I_y$ [$\unit{mV}$]} & {$\theta_1$ [$\unit{deg}$]} & {$\theta_2$ [$\unit{deg}$]} \\
    \midrule 
    45 & 82(1) & 91(1) & -98(6) & -7(6)\\
    70 & 160(2) & 150(2) & -18(1) & 72(1)\\
    120 & 100(1) & 78(1) & 23(3) & 113(3)\\
    \bottomrule
    \end{tabular}
\caption{Parametri ottenuti dal fit generale in funzione dell'angolo di rotazione dell'analizzatore.}
\label{tab:Studio lamina lambda/4 ellittica}
\end{table}


\subsubsection{Polarizzazione della luce del LED}
Si è verificato che tali valori non variano né con $\theta_{\lambda/4} = 45^\circ, \theta_a = 0^\circ$, né ruotando la lamina o l'analizzatore.
\begin{table}[H]
    \centering
    \begin{tabular}{
        l
        S[table-format=1.2(2)]
    }
        \toprule
        {Configurazione} & {$V/V_{LED}$} \\
        \midrule 
        $\lambda/4$ & 0.37(0.01) \\
        $\lambda/4$ + Analizzatore & 0.32(0.01) \\
        \bottomrule
    \end{tabular}
\caption{Intensità misurata dal fotodiodo con e senza analizzatore.}
\label{tab:Polarizzazione LED}
\end{table}

\subsubsection{Parametri di Stokes}
\begin{table}[H]
    \centering
    \begin{tabular}{
        l
        S[table-format=1.2(2)]
        S[table-format=1.2(2)]
        S[table-format=1.2(2)]
        S[table-format=1.2(2)]
    }
        \toprule
        {Polarizzazione} & {$S_0$} & {$S_1$} & {$S_2$} & {$S_3$} & {$\theta_0$ [\unit{deg}]} & {I_x} & {I_y} & DoP\\
        \midrule 
        Lineare & 1.00(0.02) & 0.97(0.02) & 0.00(0.02) & 0.00(0.02) & 44(1) & 0.98(0.02) & 0.02(0.02) \\
        \bottomrule
    \end{tabular}
\caption{Parametri di Stokes per la luce polarizzata linearmente e circolarmente}
\label{tab:Parametri di Stokes}
\end{table}









\end{document}