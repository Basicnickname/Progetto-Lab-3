\documentclass[a4paper,12pt]{article}

% --- Pacchetti Fondamentali ---
\usepackage[utf8]{inputenc} % Codifica caratteri
\usepackage[T1]{fontenc}    % Codifica font
\usepackage[italian]{babel} % Lingua italiana
\usepackage{geometry}       % Gestione margini
\usepackage{parskip}        % Gestione spazi tra paragrafi
\usepackage{hyperref}
\geometry{a4paper, margin=2.5cm}

% --- Pacchetti Matematica e Fisica ---
\usepackage{amsmath, amssymb} % Simboli matematici
\usepackage{siunitx}          % Gestione unità di misura e numeri (FONDAMENTALE)
\sisetup{
    output-decimal-marker = {.}, % Punto come separatore decimale
    separate-uncertainty = true, % Scrive errori come (Valore +/- Errore)
    per-mode = power,            % Usa / per le unità (m/s)
    inter-unit-product = \ensuremath{\cdot} % Per scrivere le unità con il punto (N·m)
}

% --- Pacchetti Immagini e Tabelle ---
\usepackage{graphicx} % Per inserire PNG/JPG
\usepackage{float}    % Per forzare la posizione delle immagini (H)
\usepackage{booktabs} % Per tabelle professionali (\toprule, \midrule)
\usepackage{caption}  % Per personalizzare le didascalie
\usepackage{subcaption}
\usepackage[font=small, labelfont=bf]{caption}
\usepackage[font=small, labelfont=bf]{subcaption}


% --- Dati Intestazione ---
\title{\textbf{Polarizzazione della luce e lamine di ritardo}}
\author{Filippo Audisio, Cataldo Insalaco, Telemaco Pezzoni}
\date{\today}

\begin{document}

\maketitle

% -------------------------------------------------------------------
\section{Obiettivo dell'esperienza}
L'obiettivo dell'esperienza è studiare il comportamento della luce polarizzata e delle lamine di ritardo, in particolare:
\begin{itemize}
    \item Verificare la legge di Malus $I(\theta)=I_0 \cos^2(\theta)$.
    \item Studiare l'utilizzo della lamina $\lambda/2$.
    \item Studiare l'utilizzo della lamina $\lambda/4$.
    \item Mostrare che la luce del LED non è polarizzata.
    \item Calcolare i parametri di Stokes per diversi stati di polarizzazione.
\end{itemize}

% -------------------------------------------------------------------
\section{Materiali e Metodi}

\subsection{Dotazione sperimentale}
\begin{itemize}
    \item LED e fotodiodo + scatola di polarizzazione.
    \item 2 lenti per la focalizzazione.
    \item 2 polarizzatori lineari.
    \item Lamine di ritardo $\lambda/2$ e $\lambda/4$.
    \item Supporti con ghiera rotante graduata.
    \item Barra di supporto.
    \item Generatore di tensione alternata.
    \item Oscilloscopio.
    \item Cavi coassiali e sdoppiatore.
\end{itemize}

\subsection{Procedura sperimentale}
Come configurazione iniziale sono stati posizionati il LED ed il fotodiodo alle estremità della barra di supporto.
Dal generatore di tensione alternata è stata sdoppiato il segnale di un'onda quadra a $\sim 6 \unit{V}$ e $\sim 1 \unit{kHz}$, il quale è stato mandato al LED per alimentarlo e ad un canale dell'oscilloscopio come trigger.
Il fotodiodo è stato polarizzato inversamente, grazie alla scatola di polarizzazione, e collegato all'altro canale dell'oscilloscopio così da poter misurare i valori di intensità trasmessa.
Sia al LED che al fotodiodo sono state applicate delle lenti per focalizzare il fascio luminoso, successivamente è stato montato un primo polarizzatore lineare sul LED ed un secondo sul supporto graduato posizionato fra LED e fotodiodo.
Infine usando il secondo polarizzatore come analizzatore si è trovato l'angolo di massima trasmissione così da misurare l'eventuale offset dovuto allo sfasamento tra i due polarizzatori. 
\subsubsection{Verifica della legge di Malus}
Dopo aver misurato il valore massimo di intensità trasmessa $V_0$, si è ruotato gradualmente l'analizzatore così da misurare $V$ in funzione di $\theta$.
\subsubsection{\texorpdfstring{Studio della lamina $\lambda/2$}{Studio della lamina lambda/2}}
Dopo aver inserito la lamina $\lambda/2$ fra i due polarizzatori e aver allineato il sistema così da misurare il massimo dell'intensità, ruotando la lamina di un angolo $\theta_1$ sono stati misurati con l'analizzatore gli angoli $\theta_2^{max}$ e $\theta_2^{min}$ corrispondenti al massimo ed al minimo di trasmissione.
\subsubsection{\texorpdfstring{Studio della lamina $\lambda/4$}{Studio della lamina lambda/4}}
Dopo aver inserito la lamina $\lambda/4$ fra i due polarizzatori 
\subsubsection{Polarizzazione della luce del LED}
Dopo aver sciolto $30 \unit{g}$ di saccarosio in acqua, sono stati aggiunti $2.5 \unit{ml}$ di HCl al $25\%$ per catalizzare l'inversione. Sono dunque stati misurati gli angoli di rotazione a intervalli di diversi minuti utilizzando luce verde; Per garantire il raggiungimento dell'equilibrio le ultime misurazioni sono state effettuate il giorno seguente. 
\subsubsection{Calcolo dei parametri di Stokes}
% -------------------------------------------------------------------
\section{Analisi dei dati e grafici}


\end{document}