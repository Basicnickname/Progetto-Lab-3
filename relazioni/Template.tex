\documentclass[11pt,a4paper]{article}

% Encoding and language
\usepackage[utf8]{inputenc}
\usepackage[T1]{fontenc}
\usepackage[english]{babel}

% Page layout
\usepackage{geometry}
\geometry{margin=25mm}
\usepackage{setspace}
% Typography & math
\usepackage{microtype}
\usepackage{amsmath,amssymb}
\usepackage{siunitx}

% Graphics, tables and captions
\usepackage{graphicx}
\usepackage{caption,subcaption}
\usepackage{booktabs}
\usepackage{float}

% Hyperlinks
\usepackage[hidelinks]{hyperref}

% Custom commands for metadata
\newcommand{\Course}{Course Name}
\newcommand{\LabTitle}{Title of the Lab Report}
\newcommand{\Authors}{Author One \\ Author Two}
\newcommand{\Supervisor}{Supervisor Name}
\newcommand{\DateSubmitted}{\today}

% Document
\begin{document}

% Title page
\begin{titlepage}
  \centering
  {\LARGE\bfseries \LabTitle \par}
  \vspace{1.5cm}
  {\large \Course \par}
  \vspace{1.0cm}
  {\large \Authors \par}
  \vfill
  {\large Supervisor: \Supervisor \par}
  \vspace{0.5cm}
  {\large Date: \DateSubmitted \par}
\end{titlepage}

\pagenumbering{roman}
\tableofcontents
\listoffigures
\listoftables
\clearpage
\pagenumbering{arabic}

% Abstract
\begin{abstract}
A short abstract (150--250 words) summarizing objectives, methods, main results, and conclusions.
\end{abstract}

\section{Introduction}
State the purpose of the experiment, background theory, and objectives. Include relevant equations and references.

Example: for a simple harmonic oscillator,
\[
m\ddot{x} + c\dot{x} + kx = 0,
\]
where \(m\) is mass, \(c\) damping coefficient and \(k\) stiffness.

\section{Materials and Methods}
Describe apparatus, materials, measurement devices, and procedures. Provide enough detail for reproducibility.

\subsection{Setup}
Include a labeled diagram or photo:
\begin{figure}[H]
  \centering
  % \includegraphics[width=0.7\textwidth]{path/to/figure.png}
  \caption{Schematic of the experimental setup. (Replace with actual image.)}
  \label{fig:setup}
\end{figure}

\subsection{Data acquisition and analysis}
Explain sampling rates, filtering, calibration, and software used. Show any data-processing equations.

\section{Results}
Present measured data with figures and tables. Use clear captions and reference them in text.

\begin{table}[H]
  \centering
  \caption{Example measurements.}
  \label{tab:example}
  \begin{tabular}{@{}l S[table-format=3.2] S[table-format=1.3]@{}}
    \toprule
    Quantity & {Value} & {Uncertainty} \\
    \midrule
    Length (mm) & 123.45 & 0.005 \\
    Time (s)    &  12.345 & 0.010 \\
    \bottomrule
  \end{tabular}
\end{table}

\begin{figure}[H]
  \centering
  % \includegraphics[width=0.8\textwidth]{path/to/plot.pdf}
  \caption{Representative result plot. Replace with actual data plot.}
  \label{fig:results}
\end{figure}

\section{Discussion}
Interpret the results, compare with theory or literature, discuss uncertainties and possible sources of error. Use propagation of uncertainty where needed:
\[
\sigma_f = \sqrt{\left(\frac{\partial f}{\partial x}\sigma_x\right)^2 + \left(\frac{\partial f}{\partial y}\sigma_y\right)^2 }.
\]

\section{Conclusion}
Summarize main findings and suggest improvements or future work.

\section*{Acknowledgments}
(Optional) Acknowledge assistance or funding.

\appendix
\section{Raw Data}
Include raw data, calibration curves, or extended derivations.

\section{Example calculations}
Show a worked example calculation used in the analysis.

% End document
\end{document}